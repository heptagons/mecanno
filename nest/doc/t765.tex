\documentclass[tikz, border=1cm]{standalone}

\usepackage{tikz}
\usetikzlibrary{calc}
\usetikzlibrary{math}

\begin{document}

\newcommand\nodes{{"B","A","C"}}
\newcommand{\meccanotriangle}[6]
{
\tikzmath{
	\x = #3; \y = #4; \z = #5;
	\xx = 1/\x; \yy = 1/\y; \zz = 1/\z;
	\cosX = (\y*\y + \z*\z - \x*\x)/(2*\y*\z); \sinX = sqrt(1 - \cosX*\cosX);
	\cosY = (\z*\z + \x*\x - \y*\y)/(2*\z*\x); \sinY = sqrt(1 - \cosY*\cosY);
	\cosZ = (\x*\x + \y*\y - \z*\z)/(2*\x*\y);
	coordinate \pX, \pY, \pZ;
	\pX = (0,0); %bottom left vertice
	\pY = (\z,0); %bottom right vertice
	\pZ = (\y*\cosX, \y*\sinX); %top vertice
	\xn = \nodes[0]; \yn = \nodes[1]; \zn = \nodes[2];
}

\begin{scope}[shift={(#1,#2)},scale={1}]

\draw[black] (\pX)
-- (\pY) foreach \t in {0,\zz,...,1} { pic [pos=\t] {code={\fill circle[radius=0.07];}}} node[midway,below]{c=\z}
-- (\pZ) foreach \t in {0,\xx,...,1} { pic [pos=\t] {code={\fill circle[radius=0.07];}}} node[midway,right]{}
-- (\pX) foreach \t in {0,\yy,...,1} { pic [pos=\t] {code={\fill circle[radius=0.07];}}} node[midway,left]{}
;
\draw[red] (\pX) node[below]{\xn};
\draw[red] (\pY) node[below]{\yn};
\draw[red] (\pZ) node[above]{\zn};
% \i/\k parses {1/1},{1/2},... where \i is the y points at left side `y' from bottom to top
% and \k is the points at right side `x' from bottom to top
\foreach \i/\k in {#6} {
	\pgfmathsetmacro\aa{int(\y - \i)}
	\pgfmathsetmacro\bb{int(\x - \k)}
	\draw[orange] (\i*\cosX, \i*\sinX) -- (\z-\k*\cosY, \k*\sinY);
	\draw[blue] (\i*\cosX, \i*\sinX) node[left]{$a_{\aa}$};
	\draw[blue] (\z-\k*\cosY, \k*\sinY) node[right]{$b_{\bb}$};
}
\end{scope}
}

\begin{tikzpicture}
\meccanotriangle{ 3}{ 0}{6}{7}{5}{{1/0},{2/1},{3/2},{4/3},{5/4},{6/5}};
\meccanotriangle{ 9}{ 0}{6}{7}{5}{{1/1},{2/2},{3/3},{4/4},{5/5}};
\meccanotriangle{15}{ 0}{6}{7}{5}{{0/1},{1/2},{2/3},{3/4},{4/5}};
\meccanotriangle{ 0}{-8}{6}{7}{5}{{0/2},{1/3},{2/4},{3/5}};
\meccanotriangle{ 6}{-8}{6}{7}{5}{{0/3},{1/4},{2/5}};
\meccanotriangle{12}{-8}{6}{7}{5}{{0/4},{1/5}};
\meccanotriangle{18}{-8}{6}{7}{5}{{0/5}};
\end{tikzpicture}

\end{document}

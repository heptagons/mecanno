\documentclass[tikz, border=1cm]{standalone}

\usepackage{tikz}
\usetikzlibrary{calc}
\usetikzlibrary{math}

\begin{document}

\newcommand{\meccanotriangle}[5]
{
\tikzmath{
	\x = #1; \y = #2; \z = #3;
	\xx = 1/\x; \yy = 1/\y; \zz = 1/\z;
	\cosX = (\y*\y + \z*\z - \x*\x)/(2*\y*\z); \sinX = sqrt(1 - \cosX*\cosX);
	\cosY = (\z*\z + \x*\x - \y*\y)/(2*\z*\x); \sinY = sqrt(1 - \cosY*\cosY);
	\cosZ = (\x*\x + \y*\y - \z*\z)/(2*\x*\y);
	coordinate \pX, \pY, \pZ;
	\pX = (0,0); %bottom left vertice
	\pY = (\z,0); %bottom right vertice
	\pZ = (\y*\cosX, \y*\sinX); %top vertice
}

\begin{scope}[shift={(0,0)},scale={1}]

\draw[black] (\pX)
-- (\pY) foreach \t in {0,\zz,...,1} { pic [pos=\t] {code={\fill circle[radius=0.07];}}} node[midway,below]{\z}
-- (\pZ) foreach \t in {0,\xx,...,1} { pic [pos=\t] {code={\fill circle[radius=0.07];}}} node[midway,right]{\x}
-- (\pX) foreach \t in {0,\yy,...,1} { pic [pos=\t] {code={\fill circle[radius=0.07];}}} node[midway,left]{\y}
;
\foreach \i/\k [count=\j] in {#4} { % \i/\k parses 1/1,1/2,...!
	\draw[green] (-1,\j) node {(\i) debug (\k)};
	\draw[red] (\i*\cosX, \i*\sinX) -- (\z-\k*\cosY, \k*\sinY);
}
\end{scope}
}

\begin{tikzpicture}
\meccanotriangle{6}{7}{5}{{1/0},{2/1},{3/2},{4/3},{5/4},{6/5}};
\end{tikzpicture}

\end{document}

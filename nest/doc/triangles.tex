\documentclass[11pt]{article}
\title{Meccano triangles}
\author{https://github.com/heptagons/meccano/nest}
\date{}

\newfam\bbbfam
\font\bbbten=msbm10
\font\bbbseven=msbm7
\font\bbbfive=msbm5
\textfont\bbbfam=\bbbten
\scriptfont\bbbfam=\bbbseven
\scriptscriptfont\bbbfam=\bbbfive
\def\bbb{\fam=\bbbfam}

\usepackage{../../meccano}
\begin{document}

\maketitle
\begin{abstract}
We construct meccano triangles. Basic triangles has the three sides as integers and calculate the internal diagonal distances.
Such diagonals then are used as the new side of more complicated triangles and then again we
calculate new distances formed and so on. Eventually we expect to
find certain angles joining the triangles which can be used to construct regular polygons or more figures.
\end{abstract}

\section{Triangle $(a,b,c)$}
A triangle $(a,b,c)$ has the tree sides $a$, $b$ and $c$ where $a,b,c \in \bbb N$. To avoid repetitions we
consider only the cases:
\begin{align}
a &\ge b \ge c\\
a &< b + c
\end{align}

\subsection{Triangle $(a,b,c)$ diagonals}

To calculate the diagonals from side $a$ to side $b$ we start calculating $\cos{C}$ where $C$ is the
opposite angle of side $c$:
\begin{align}
\cos{C} &= \frac{a^2 + b^2 - c^2}{2ab}
\end{align}

Then with the $\cos{C}$ we can calculate every diagonal $\overline{a_xb_y}$ with
the law of cosines:
\begin{align}
\overline{a_xb_y} &= \sqrt{x^2 + y^2 - 2xy\cos{C}}\\
       &= \sqrt{x^2 + y^2 - 2xy\frac{a^2 + b^2 - c^2}{2ab}}\\
       &= \frac{\sqrt{a^2b^2(x^2 + y^2)-abxy(a^2 + b^2 - c^2)}}{ab}
\end{align}

where $1 \le x \le a$, $1 \le y \le b$ and $x - y \ge 0$.

By inspection we deduce that for basic meccano triangles:
\begin{align}
a, b, c &\in \bbb{N}\\
\cos{A}, \cos{B}, \cos{C} &\in \bbb{Q}\\
\overline{a_xb_y}, \overline{b_yc_z}, \overline{a_xc_z} &\in \bbb{A}
\end{align}

\subsection{Example triangle [7,6,5]}

\begin{figure}[htp]
\centering
\includegraphics[scale=1]{t765bc}
\caption{Triangle $[7,6,5]$, $bc$ diagonals.}
\label{t765bc}
\end{figure}

\begin{figure}[htp]
\centering
\includegraphics[scale=1]{t765ac}
\caption{Triangle $[7,6,5]$, $ac$ diagonals.}
\label{t765ac}
\end{figure}

\begin{figure}[htp]
\centering
\includegraphics[scale=1]{t765ab}
\caption{Triangle $[7,6,5]$, $ab$ diagonals.}
\label{t765ab}
\end{figure}


Figure \ref{t765bc} show triangle $[7,6,5]$ diagonals $b_yc_z$.
\\\\
Figure \ref{t765ac} show triangle $[7,6,5]$ diagonals $a_xc_z$.
The diagonals join points from side $a$ nodes described as $a_x$ to side $c$ nodes described as $c_z$ in all combinations.
\\\\
Figure \ref{t765ab} show triangle $[7,6,5]$ diagonals $a_xb_y$.
The diagonals join points from side $a$ nodes described as $a_x$ to side $b$ nodes described as $b_y$ in all combinations.
\\\\
\newcommand\five{\colorbox{green}{$5$}}
Matrix for angle $A$ diagonals joining sides $b$ and $c$. Empty cells are repetitions:
\begin{equation}\label{eq:appendrow}
\left(\begin{array}{cccccc}
	\frac{2\sqrt{10}}{5} & \frac{\sqrt{105}}{5} & \frac{2\sqrt{55}}{5} & \frac{\sqrt{385}}{5} & 2\sqrt{6} & \frac{\sqrt{865}}{5} \\
	& \frac{4\sqrt{10}}{5} & \frac{\sqrt{265}}{5} & \frac{2\sqrt{105}}{5} & \five & \frac{4\sqrt{55}}{5} \\
	& & \frac{6\sqrt{10}}{5} & \frac{\sqrt{505}}{5} & 2\sqrt{7} & \frac{3\sqrt{105}}{5}\\
	& & & \frac{8\sqrt{10}}{5} & \sqrt{33} & \frac{2\sqrt{265}}{5}\\
	& & & & 2\sqrt{10} & \boxed{7} \\
\end{array}\right)
\end{equation}


Matrix for angle $B$ diagonals joining sides $a$ and $c$. Empty cells are repetitions.
Values at column 7 (at the right of separator $|$) are repeated and already in previous matrix.
\begin{equation}\label{eq:appendrow}
\left(\begin{array}{cccccccc}
	\frac{4\sqrt{70}}{35} & \frac{3\sqrt{385}}{35} & \frac{2\sqrt{2065}}{35} & \frac{\sqrt{15505}}{35} & \frac{12\sqrt{7}}{7} & \frac{\sqrt{37345}}{35} & | & \frac{2\sqrt{265}}{5}\\
	& \frac{8\sqrt{70}}{35} & \frac{\sqrt{7945}}{35} & \frac{6\sqrt{385}}{35} & \frac{\sqrt{889}}{7} & \frac{4\sqrt{2065}}{35} & | & \frac{3\sqrt{105}}{5} \\
	& & \frac{12\sqrt{70}}{35} & \frac{\sqrt{14665}}{35} & \frac{2\sqrt{217}}{7} & \frac{9\sqrt{385}}{35} & | & \frac{4\sqrt{55}}{5}\\
	& & & \frac{16\sqrt{70}}{35} & \frac{3\sqrt{105}}{7} & \frac{2\sqrt{7945}}{35} & | & \frac{\sqrt{865}}{5}\\
	& & & & \frac{4\sqrt{70}}{7} & \frac{\sqrt{1393}}{7} & | & \boxed{6}\\
\end{array}\right)
\end{equation}

Matrix for angle $C$ diagonals joining sides $a$ and $b$. Empty cells are repetitions.
Values at columns 6 and 7 (at the right of separator $|$) are repeated and already in previous matrices.
\begin{equation}\label{eq:appendrow}
\left(\begin{array}{cccccccc}
	\frac{2\sqrt7}7 & \frac{\sqrt{105}}7 & \frac{2\sqrt{70}}7 & \frac{\sqrt{553}}7 & \frac{2\sqrt{231}}7 & | &  \frac{\sqrt{1393}}7 & 2\sqrt{10} \\
	 & \frac{4\sqrt7}7 & \frac{\sqrt{217}}7 & \frac{2\sqrt{105}}7 & \frac{\sqrt{721}}7 & | &  \frac{4\sqrt{70}}7 & \sqrt{33} \\
	 & & \frac{6\sqrt7}7 & \frac{\sqrt{385}}7 & \frac{2\sqrt{154}}7 & | &  \frac{3\sqrt{105}}7 & 2\sqrt{7} \\
	 & & & \frac{8\sqrt7}7 & \frac{\sqrt{609}}7 & | &  \frac{2\sqrt{217}}7 & \five \\
	 & & & & \frac{10\sqrt7}7 & | &  \frac{\sqrt{889}}7 & 2\sqrt{6} \\
	 & & & & & | & \frac{12\sqrt7}7 & \boxed{5} \\
\end{array}\right)
\end{equation}

\section{Triangles$(\sqrt{a},b,c)$}

Triangles$(\sqrt{a},b,c)$ have the tree sides $\sqrt{a}$, $b$ and $c$ where $a^2,b,c \in \bbb N$. So we have:
\begin{align}
\sqrt{a} &\ge b \ge c\\
      a  &\ge b^2 \ge c^2
\end{align}
And:
\begin{align}
\sqrt{a} &< b + c\\
a &< (b + c)^2
\end{align}

\subsection{Example triangles$(2\sqrt{6},b,c)$}

In this case $\sqrt{a} = 2\sqrt{6}$ so $a = 24$. Then $b = \{ 1,2,3,4 \}$ because $b^2 = \{ 1,4,9,16\} < 24$.
Also $c = \{ 1,2,3,4 \}$ because $c^2 = \{ 1,4,9,16\} < 24$. We form a matrix with with the values $(b+c)^2$:
\begin {equation}\label{E:1}
(b_i + c_j)^2 =\bordermatrix{~ & b=1 & b=2 & b=3 & b=4 \cr
c=1 &  2 &  9 & 16 & 25 \cr    
c=2 &  9 & 16 & 25 & 36 \cr    
c=3 & 16 & 25 & 36 & 49 \cr    
c=4 & 25 & 36 & 49 & 64 \cr}
    \end {equation}

Then we remove cells which don't fulfil condition $b \ge c$:
\begin {equation}\label{E:2}
(b_i + c_j)^2 =\bordermatrix{~ & b=1 & b=2 & b=3 & b=4 \cr
c=1 &  2 &  9 & 16 & 25 \cr    
c=2 &    & 16 & 25 & 36 \cr    
c=3 &    &    & 36 & 49 \cr    
c=4 &    &    &    & 64 \cr}
    \end {equation}

Then we remove cells which don't fulfil condition $a < (b+c)^2$:
\begin {equation}\label{E:3}
(b_i + c_j)^2 =\bordermatrix{~ & b=1 & b=2 & b=3 & b=4 \cr
c=1 &    &    &    & 25 \cr    
c=2 &    &    & 25 & 36 \cr    
c=3 &    &    & 36 & 49 \cr    
c=4 &    &    &    & 64 \cr}
    \end {equation}
    
So only six triangles are valid: $(2\sqrt{6},4,1)$, $[2\sqrt{6},3,2]$, $[2\sqrt{6},4,2]$, $[2\sqrt{6},3,3]$, $[2\sqrt{6},4,3]$ and $[2\sqrt{6},4,4]$.

\end{document}
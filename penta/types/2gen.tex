\documentclass[tikz, border=1cm]{standalone}

\usepackage{tikz}
\usetikzlibrary{calc}
\usetikzlibrary{math}

\begin{document}

\begin{tikzpicture}
\newcommand{\gen}[6]
{
\tikzmath{
	\cosA = cos(72); \sinA = sin(72);
	\cosB = cos(36); \sinB = sin(36);
	\a = #1; \b = #2; \c = #3; \d = #4; \e = #5; \f = #6;
	%reds
	coordinate \pa, \pb, \pc, \pd, \pe, \pf, \pg, \ph, \pI;
	int \nab, \nbc, \ncd, \nde, \nef, \nfg, \ngh, \nHI;
	\pa = (0,0); \pd = (\a,0);
	if \a > 2*\c then {
		\pb = (\c,0); \pc = (\a-\c,0);
		\nab = \c; \nbc = \a - 2*\c; \ncd = \c;
	} else {
		\pb = (\a-\c,0); \pc = (\c,0);
		\nab = \a-\c; \nbc = 2*\c - \a; \ncd = \a-\c;
	};
	\nde = \a;    \pe = (\pd) + ( \nde*\cosA,  \nde*\sinA);
	\nef = \b;    \pf = (\pe) + (-\nef*\cosB,  \nef*\sinB); 
	\nfg = \a-\b; \pg = (\pf) + (-\nfg*\cosB,  \nfg*\sinB);
	\ngh = \a-\b; \ph = (\pg) + (-\nfg*\cosB, -\nfg*\sinB);
	\nHI = \b;    \pI = (\ph) + (-\nHI*\cosB, -\nHI*\sinB); 
}

\begin{scope}[scale={\f}]
	\draw[red] (\pa)
	-- (\pb) node [midway,below]{\nab}
	-- (\pc) node [midway,below]{\nbc}
	-- (\pd) node [midway,below]{\ncd}
	-- (\pe) node [midway]{\nde}
	-- (\pf) node [midway]{\nef}
	-- (\pg) node [midway]{\nfg}
	-- (\ph) node [midway]{\ngh}
	-- (\pI) node [midway]{\nHI}
	-- cycle node [midway]{\a};
	\foreach \p in {\pa, \pb, \pc, \pd, \pe, \pf, \pg, \ph, \pI} {
		\fill[red] (\p) circle [radius=3pt];
	};
\end{scope}

 
 
}

%\gen{12}{2}{9}{6}{11}{0.5};
\gen{12}{6}{3}{10}{11}{0.5};

\end{tikzpicture}
\end{document}

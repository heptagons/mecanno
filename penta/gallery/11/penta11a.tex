\documentclass[tikz, border=1cm]{standalone}

\usepackage{tikz}
\usetikzlibrary{calc}
\usetikzlibrary{math}
\usepackage{../../../meccano}
\usepackage{../penta}

\newcommand{\meccanofivesurd}[8]{ %x,y,z, p, color1,c2,c3,c4,c5
 \pgfmathsetmacro\x{#1}
 \pgfmathsetmacro\xa{abs(#1)}
 \pgfmathsetmacro\y{#2}
 \pgfmathsetmacro\yy{2*\y}
 \pgfmathsetmacro\z{#3}
 \pgfmathsetmacro\aa{-asin(\x/\yy)}
 \pgfmathsetmacro\ab{-acos(\x/\yy)}
 \begin{scope}[rotate=90+\aa]
  \meccanostrip[#5]{\yy}{1}{#4} 
  \path (\y,0) ++(90:20+#4) node{$H$};
  \path (\yy,0) ++(0:20+#4) node{$I$};
  \begin{scope}[shift={(\y,0)},rotate=-2*\aa]
   \meccanostrip[#6]{\y}{1}{#4}
   \path (\y,0) ++(-60:20+#4) node{$J$};
  \end{scope}
  \begin{scope}[shift={(\yy,0)},rotate=\ab]
   \meccanostrip[#7]{\xa}{1}{#4}
   \begin{scope}[shift={(\xa,0)},rotate=90]
    \meccanostrip[#8]{\z}{1}{#4}
   \end{scope}
   \begin{scope}[shift={(\xa-4,0)},rotate=atan(3/4)]
    \meccanostrip[888888]{5}{1}{#4} %pythagoras 3-4-5 diagonal
    \path (5,0) ++(-30:20+#4) node{$K$};
   \end{scope}
  \end{scope}
 \end{scope}
}

\begin{document}
 \begin{tikzpicture}
  \begin{scope}[scale={0.5}]
   \pgfmathsetmacro\s{11}
   %\pgfmathsetmacro\abscissa{(19-5*sqrt(5))/4}
   %\pgfmathsetmacro\ordinate{sqrt(1450+190*sqrt(5))/4}
   %\pgfmathsetmacro\angle{atan(\ordinate/\abscissa)}
   \def\p{5pt}
   \meccanopentagon{FF4400}{\s}{\p}{A}{B}{C}{D}{E}
   % AB
   \begin{scope}[shift={(3,0)},rotate=-47.5]%TODO
    \meccanofivesurd{-4}{3}{11}{\p}{0000FF}{008800}{880088}{FF4400}
    \path (0,0) ++(-30:20+\p) node{$F$};
   \end{scope}
   \begin{scope}[shift={(\s,0)},rotate=72]
    %BC
    \begin{scope}[shift={(\s,0)},rotate=72]
     %CD
     \begin{scope}[shift={(\s,0)},rotate=72]
      %DE
      \begin{scope}[shift={(3,0)},rotate=-47.5]%TODO
       \meccanofivesurd{-4}{3}{11}{\p}{0000FF}{008800}{880088}{FF4400}
       \path (0,0) ++(-30:20+\p) node{$G$};
      \end{scope}
      \begin{scope}[shift={(\s,0)},rotate=72]
       %EA
      \end{scope}
     \end{scope}
    \end{scope}
   \end{scope}
  \end{scope}
 \end{tikzpicture}
\end{document}
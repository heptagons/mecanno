\documentclass[11pt]{article}
\title{Meccano pentagons gallery}
\author{https://github.com/heptagons/meccano/penta/gallery}
\date{}

\usepackage{../../meccano}
\begin{document}

\maketitle
\begin{abstract}
We show constructions of meccano rigid regular pentagons from side $12$ to $3$. We restrict all internal strips, we call diagonals, to remain inside the pentagon's perimeter and that don't overlap.
Several programs found the solutions and we show some alternatives and prove the claimed values are exact.
\end{abstract}

\section{Pentagons of size 12}

A program found that side 12 is the smallest pentagon that can be made rigid with a rhoumbus and two strips as diagonals so need only 4 strips as diagonals. We show two cases.

\begin{figure}[h]
 \centering
 \includegraphics[scale=1]{12/penta12a}
 \caption{Pentagon size 12 (case a).}
 \label{fig:penta12a}
\end{figure}

Figure \ref{fig:penta12a} show a regular pentagon $A,B,C,D,E$ of side 12 with a rhombus $D,I,H,G$ of side $9$. We prove strips $AJ,BF$ are correct. First we calculate the abscissas going through vertices $A,E,I,J$ substracting when we move to the left and adding when we move to the right:
\begin{align}
AJ_x &= AE_x + EI_x + IJ_x\\
 &= -\overline{AE}\cos\left(\frac{2\pi}5\right)
 + \overline{EI}\cos\left(\frac{\pi}5\right) 
 + \overline{IJ}\cos\left(\frac{\pi}5\right)\nonumber\\
 &= -12\left(\frac{\sqrt5 - 1}4\right)
  +3\left(\frac{1+\sqrt5}4\right)
  +4\left(\frac{1+\sqrt5}4\right) = \frac{19-5\sqrt5}4
\end{align}

Then we calculate the ordinates going to the same order of vertices adding when we go up and substracting when we go down:
\begin{align}
AJ_y &= -AE_y + EI_y + IJ_y\\
 &= \overline{AE}\sin\left(\frac{2\pi}5\right)
 + \overline{EI}\sin\left(\frac{\pi}5\right) 
 - \overline{IJ}\sin\left(\frac{\pi}5\right)\nonumber\\
 &= 12\left(\frac{\sqrt{10+2\sqrt5}}4\right)
 + 3\left(\frac{\sqrt{10-2\sqrt5}}4\right)
 - 4\left(\frac{\sqrt{10-2\sqrt5}}4\right)\nonumber\\
 &= \frac{12\sqrt{10+2\sqrt5} - \sqrt{10-2\sqrt5}}4 = \frac{\sqrt{1450+190\sqrt5}}4
\end{align}
Finally we calculate the distance $\overline{AJ}$ which coincides with strip size $11$:
\begin{align}
\overline{AJ} &= \sqrt{(AJ_x)^2 + (AJ_y)^2}\\
 &= \sqrt{\left(\frac{19-5\sqrt5}4\right)^2 + \frac{1450+190\sqrt5}{16}}\nonumber\\
 &= \sqrt{\frac{486-190\sqrt5}{16} + \frac{1450+190\sqrt5}{16}} = \sqrt{121} = 11
\end{align}

\begin{figure}[h]
 \centering
 \includegraphics[scale=1]{12/penta12b}
 \caption{Pentagon size 12 case (b).}
 \label{fig:penta12b}
\end{figure}

Figure \ref{fig:penta12b} show a regular pentagon $A,B,C,D,E$ of size 12 with a rhombus $D,I,H,G$ of size $12$. We prove strips $GH,IJ$ are correct. First we calculate the abscissas going through vertices $G,A,E,H$ substracting when we move to the left and adding when we move to the right:
\begin{align}
GH_x &= -GA_x - AE_x + EH_x\\
 &= -\overline{GA} - \overline{AE}\cos\left(\frac{2\pi}5\right)
 +\overline{EH}\cos\left(\frac{\pi}5\right)\nonumber\\
 &= -4 - 12\left(\frac{\sqrt5 - 1}4\right) + 3\left(\frac{1+\sqrt5}4\right)
 = \frac{-1-9\sqrt5}4
\end{align}

Then we calculate the ordinates going to the same order of vertices adding when we go up and substracting when we go down:
\begin{align}
GH_y &= AG_y + AE_y - EH_y\\
 &= 0 + \overline{AE}\sin\left(\frac{2\pi}5\right)
 - \overline{EH}\sin\left(\frac{\pi}5\right)\nonumber\\
 &= 12\left(\frac{\sqrt{10+2\sqrt5}}4\right)
 - 3\left(\frac{\sqrt{10-2\sqrt5}}4\right)\nonumber\\
 &= \frac{12\sqrt{10+2\sqrt5} -3\sqrt{10-2\sqrt5}}4 = \frac{\sqrt{1530-18\sqrt5}}4
\end{align}

Finally we calculate the distance $\overline{GH}$ which coincides with strip size $11$:
\begin{align}
\overline{GH} &= \sqrt{(GH_x)^2 + (GH_y)^2}\\
 &= \sqrt{\left(\frac{-1-9\sqrt5}{4}\right)^2 + \frac{1530-18\sqrt5}{16}}\nonumber\\
 &= \sqrt{\frac{406+18\sqrt5}{16} + \frac{1530-18\sqrt5}{16}} = \sqrt{121} = 11
\end{align}


\section{Pentagon of size 11}

\end{document}
\documentclass[11pt]{article}
\title{\textbf{Meccano four frame}}
\author{https://github.com/heptagons/meccano/frames/four}
\date{}

\usepackage{../../meccano}
\usepackage{../frames}
%\usepackage{tikz}
%\usetikzlibrary{calc}
%\usepackage{lipsum}

%\newenvironment{MyColorPar}[1]{
%    \leavevmode\color{#1}\ignorespaces
%}{
%}

\begin{document}

\maketitle
\begin{abstract}
Four frame is a group of four rigid meccano\meccanoref strips.
\end{abstract}

\section{Four frame}

\begin{figure}[H]
 \centering
 \begin{tikzpicture}
  \begin{scope}[scale=1]
   \meccanoframefour{4}{2}{3}{2}{3}{3pt}
  \end{scope}
 \end{tikzpicture}
 \caption{Antisymmetric four frame.}
 \label{fig:four}
\end{figure}

Figure \ref{fig:four} show the antisymmetric four-strips frame.
From the figure we define $\alpha \equiv \angle{BA_1C_1}$ and define integers $m=b^2 + c^2 - a^2$ and $n=2bc$ using the law of cosines, then we calculate $\cos\alpha$ and $\sin\alpha$:
\begin{align}
(\alpha,m,n) &\equiv (\angle{BA_1C_1},b^2 + c^2 - a^2, 2bc)\\
\cos\alpha &= \frac{m}n\\
\sin\alpha &= \frac{\sqrt{n^2-m^2}}n
\end{align}

We calculate the distance $f = \overline{C_2D}$ with the law of cosines using angle $\alpha$
and defining integers $x = a+c$ and $y = b+d$:
\begin{align}
x &\equiv a + c\\
y &\equiv b + d\\
f^2 &= (a+c)^2 + (b+d)^2 - 2(a+c)(b+d)\cos\alpha\\
 &= x^2 + y^2 - \frac{2mxy}n\\
f &= \frac{\sqrt{n^2(x^2 + y^2) - 2mnxy}}n
\end{align}

We define a new integer $z \equiv n^2(x^2 + y^2) - 2mnxy$ so we have:
\begin{align}
z &\equiv n^2(x^2 + y^2) - 2mnxy\\
f &= \frac{\sqrt{z}}n
\end{align}

We define angle $\theta \equiv \angle{A_1C_2D}$ and calculate $\cos\theta$ and $\sin\theta$:
\begin{align}
\theta &\equiv \angle{A_1C2D}\\
\cos\theta &= \frac{(a+c)^2 + f^2 - (b+d)^2}{2(a+c)f}\nonumber\\
 &= \frac{x^2 + f^2 - y^2}{2xf}\nonumber\\
 &= \frac{x^2 + x^2 + y^2 - \dfrac{2mxy}{n} - y^2}{2x\dfrac{\sqrt{z}}{n}}\nonumber\\
 &= \frac{nx - my}{\sqrt{z}} \equiv \frac{o}p\\
\sin\theta &= \frac{\sqrt{p^2 - o^2}}p\nonumber\\
 &= \frac{\sqrt{n^2(x^2 + y^2) - 2mnxy - (nx-my)^2}}p\nonumber\\
 &= \frac{\sqrt{n^2(x^2 + y^2) - 2mnxy - n^2x^2 +2nxmy - m^2y^2}}p\nonumber\\
 &= \frac{\sqrt{n^2y^2 - m^2y^2}}p \equiv \frac{q}{p}
\end{align}

From the figure \ref{fig:four} we define $\gamma \equiv \angle{BA_2C_2}$ and define integers $s=a^2 + b^2 - c^2$ and $t=2ab$ and calculate $\cos\gamma$ and $\sin\gamma$:
\begin{align}
(\gamma,s,t) &\equiv (\angle{BA_2C_2}, a^2 + b^2 - c^2, 2ab)\\
\cos\gamma &= \frac{s}{t}\\
\sin\gamma &= \frac{\sqrt{t^2-s^2}}{t}
\end{align}

We define angle $\phi \equiv \angle{A_2C_2D}$ and we note is the sum of angles $\theta + \gamma$ and we calculate $\cos\phi$:
\begin{align}
\phi &\equiv \angle{A_2C_2D}\\
 &= \theta + \gamma\\
\cos\phi &= \cos(\theta + \gamma)\\
 &= \cos\theta\cos\gamma - \sin\theta\sin\gamma\nonumber\\
 &= \frac{os}{pt} - \frac{q\sqrt{t^2-s^2}}{pt}\nonumber\\
 &= \frac{os - q\sqrt{t^2-s^2}}{pt}
\end{align}

From the figure we define angle $\psi \equiv \angle{DC_2E}$ and we note equals angle $\pi - \phi$, so we have:
\begin{align}
\psi &\equiv \angle{DC_2E}\\
 &= \pi - \phi\\
\cos\psi &= \cos(\pi - \phi)\\
 &= -\cos\phi\nonumber\\
 &= \frac{q\sqrt{t^2-s^2} - os}{pt}
\end{align}

Finally with $\cos\psi$, $e$ and $f$ we can calculate distance $g = \overline{ED}$:
\begin{align}
g^2 &= e^2 + f^2 - 2ef\cos\psi\\
 &= e^2 + x^2 + y^2 - \frac{2mxy}{n} 
 - 2e\left(\frac{\sqrt{z}}{n}\right)\left(\frac{q\sqrt{t^2-s^2} - os}{\sqrt{z}t}\right)\\
\end{align}

%\subsection{Using $beta$}
%
%From the figure we define $\beta \equiv \angle{A_1BC_1}$ and define integers $m=a^2 + c^2 - b^2$ and $n=2ac$ using the law of cosines, then we calculate $\cos\beta$ and $\sin\beta$:
%\begin{align}
%(\beta,m,n) &\equiv (\angle{A_1BC_1},a^2 + c^2 - b^2, 2ac)\\
%\cos\beta &= \frac{m}n\\
%\sin\beta &= \frac{\sqrt{n^2-m^2}}n
%\end{align}
%
%We calculate distance $f_0 = \overline{C_1C_2}$ with the law of cosines using angle $\pi - \beta$:
%\begin{align}
%\pi - \beta &= \angle{C_1BC_2}\\
%f_0^2 &= a^2 + a^2 - 2aa\cos(\pi-\beta)\\
% &= 2a^2(1 + \cos\beta)\nonumber\\
% &= \frac{2a^2(m+n)}n\nonumber\\
%f_0 &= \frac{a\sqrt{2n(m+n)}}{n}, \qquad m + n = \frac{f_0^2n}{2a^2}
%\end{align}
%
%From the figure we define $\theta \equiv \angle{A_1C_1C_2}$ and calculate $\cos\theta$ in function of $f_0$ using the law of cosines:
%\begin{align}
%\theta &\equiv \angle{A_1C_1C_2}\\
%\cos\theta &= \frac{b^2 + f_0^2 - (a+c)^2}{2bf_0}\\
% &= \frac{f_0^2 - (a^2 + 2ac + c^2 - b^2)}{2bf_0}\nonumber\\
% &= \frac{f_0^2 - (m + n)}{2bf_0}\nonumber\\
% &= \frac{f_0^2 - \dfrac{f_0^2n}{2a^2}}{2bf_0}\nonumber\\
% &= \frac{(2a^2 - n)f_0}{4a^2b}\nonumber\\
% &= \frac{(a - c)f_0}{2ab}
%\end{align}
%
%Now we calculate distance $f_1$ with the law of cosines using angle $\pi-\theta$:
%\begin{align}
%\pi - \theta &\equiv \angle{C_1C_2D}\\
%f_1^2 &= f_0^2 + d^2 - 2f_0d\cos(\pi-\theta)\\
% &= f_0^2 + d^2 + 2f_0d\cos\theta\nonumber\\
% &= f_0^2 + d^2 + 2f_0d\frac{(a - c)f_0}{2ab}\nonumber\\
% &= f_0^2 + d^2 + \frac{ad - cd}{ab}f_0^2\nonumber\\
% &= \frac{ab + ad - cd}{ab}f_0^2 + d^2\nonumber\\
% &= \frac{ab + ad - cd}{ab}\left(\frac{2a^2(m+n)}n\right) + d^2\nonumber\\
% &= \frac{2a(m+n)(ab + ad - cd)}{bn} + d^2\nonumber\\
% &= \frac{(m+n)(ab + ad - cd)}{bc} + d^2\nonumber\\
% &= \frac{\sqrt{bc(m+n)(ab + ad - cd) + b^2c^2d^2}}{bc}
%\end{align}
%
%From the figure \ref{fig:four} we define $\delta \equiv \angle{A_1C_2D}$ and calculate $\cos\delta$ and $\sin\delta$:
%\begin{align}
%\delta &\equiv \angle{A_1C_2D}\\
%\cos\delta &= \frac{(a+c)^2 + f_1^2 - (b+d)^2}{2(a+c)f_1}\nonumber\\
% &= \frac{a^2 + 2ac + c^2 - b^2 + f_1^2 - 2bd - d^2}{2(a+c)f_1}\nonumber\\
% &= \frac{m + n + f_1^2 - 2bd - d^2}{2(a+c)f_1}\nonumber\\
%\end{align}
%
%
%%We notice that $\gamma \equiv \angle{A_2C_2A1}$, $\gamma + \delta = \angle{A_2C_2D}$
%%and $\epsilon \equiv \angle{DC_2E} = \pi - (\gamma + \delta)$:



\end{document}

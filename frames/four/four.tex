\documentclass[11pt]{article}
\title{\textbf{Meccano four frame}}
\author{https://github.com/heptagons/meccano/frames/four}
\date{}

\usepackage{../../meccano}
\usepackage{../frames}
%\usepackage{tikz}
%\usetikzlibrary{calc}
%\usepackage{lipsum}

%\newenvironment{MyColorPar}[1]{
%    \leavevmode\color{#1}\ignorespaces
%}{
%}

\begin{document}

\maketitle
\begin{abstract}
Four frame is a group of four rigid meccano\meccanoref strips.
\end{abstract}

\section{Four frame}

\begin{figure}[H]
 \centering
 \begin{tikzpicture}
  \begin{scope}[scale=1]
   \meccanoframefour{4}{2}{3}{2}{3}{3pt}
  \end{scope}
 \end{tikzpicture}
 \caption{Four frame.}
 \label{fig:four}
\end{figure}

Figure \ref{fig:four} show the four-strips frame. First we calculate $f_0$ with the law of cosines:
\begin{align}
\beta &\equiv \angle{A_1BC_1}\\
\cos\beta &= \frac{a^2 + c^2 - b^2}{2ac}\\
\pi - \beta &= \angle{C_1BC_2}\\
f_0^2 &= a^2 + a^2 - 2aa\cos(\pi-\beta)\\
 &= 2a^2(1 + \cos\beta)\nonumber\\
 &= 2a^2\frac{2ac + a^2 + c^2 - b^2}{2ac} = \frac{a(2ac + a^2 + c^2 - b^2)}c\nonumber\\
 &= \frac{a((a + c)^2 - b^2)}c = \frac{a(a + c + b)(a + c - b)}c \nonumber\\
f_0 &= \frac{\sqrt{ac(a+b+c)(a-b+c)}}c
\end{align}
To simplify we define $m = c(a+c+b)(a+c-b)$ so we have:
\begin{align}
m &\equiv c(a+b+c)(a-b+c)\\
f_0 &= \frac{\sqrt{am}}c
\end{align}

We first found $\cos\theta$ in function of $f_0$ also with the law of cosines:
\begin{align}
\theta &\equiv \angle{A_1C_1C_2}\\
\cos\theta &= \frac{b^2 + f_0^2 - (a+c)^2}{2bf_0}\\
 &= \frac{f_0^2 + (b^2 - (a+c)^2)}{2bf_0}\nonumber\\
 &= \frac{f_0^2 + (b+a+c)(b-a-c)}{2bf_0}\nonumber\\
 &= \frac{f_0^2 - (a+b+c)(a-b+c)}{2bf_0}\nonumber\\
 &= \frac{f_0^2 - \dfrac{m}c}{2bf_0}\nonumber\\
 &= \frac{f_0^2 - \dfrac{f_0^2c^2}{ac}}{2bf_0}\nonumber\\
 &= \frac{f_0 - \dfrac{f_0c}{a}}{2b}\nonumber\\
 &= \frac{(a-c)f_0}{2ab}
\end{align}

Now we calculate $f_1$:
\begin{align}
\pi - \theta &\equiv \angle{C_1C_2D}\\
f_1^2 &= f_0^2 + d^2 - 2f_0d\cos(\pi-\theta)\nonumber\\
 &= f_0^2 + d^2 + 2f_0d\cos\theta\nonumber\\
 &= f_0^2 + d^2 + 2f_0d\frac{(a-c)f_0}{2ab}\nonumber\\
 &= f_0^2 + d^2 + \frac{d(a-c)}{ab}f_0^2\nonumber\\
 &= d^2 + \frac{ab + d(a-c)}{ab}f_0^2\nonumber\\
 &= d^2 + \frac{ab + d(a-c)}{ab}\left(\frac{am}{c^2}\right)\nonumber\\
 &= d^2 + \frac{m(ab + d(a-c))}{bc^2}\nonumber\\
f_1 &= \frac{\sqrt{b^2c^2d^2 + bm(ab + d(a-c))}}{bc}
\end{align}

\end{document}

\documentclass[11pt]{article}
\title{\textbf{Meccano frames}}
\author{https://github.com/heptagons/meccano/frames}
\date{}

\usepackage{../meccano}
\usepackage{tikz}
\usetikzlibrary{calc}

\newcommand{\meccanoframetriangle}[7]{%a,b,c,d,e,p,scale
\begin{tikzpicture}
\def\p{#6}
\pgfmathsetmacro{\ad}{#1 + #4} % a+d
\pgfmathsetmacro{\be}{#2 + #5} % b+e
\pgfmathsetmacro{\cosb}{(#1^2 + #3^2 - #2^2)/(2*#1*#3)} % (a^2 + c^2 - b^2)/(2ac)
\pgfmathsetmacro{\cosa}{(#2^2 + #3^2 - #1^2)/(2*#2*#3)} % (b^2 + c^2 - a^2)/(2bc)
\begin{scope}[scale=#7]
  \meccanostrip[0000cc]{#3}{1}{\p} % blue c=BA
   \begin{scope}[rotate={acos(\cosb)},shift={(-#4,0)}] % shift -d
    \meccanostrip[cc0000]{\ad}{1}{\p} % red a+d=CB+BE
    \path (0,0) ++(180:5*\p) node{$E$};
    \path (#4,0) ++(90:5*\p) node{$B$}; %d
    \path (\ad,0) ++(90:5*\p) node{$C$};
   \end{scope}
   \begin{scope}[shift={(#3,0)},rotate=180-acos{\cosa},shift={(-#5,0)}] % shift c, shift -e
    \meccanostrip[008800]{\be}{1}{\p} % green b+e=CA+AD
    \path (#5,0) ++(-90:5*\p) node{$A$}; %e
    \path (0,0) ++(-180:5*\p) node{$D$};
   \end{scope}
\end{scope}
\end{tikzpicture}
}

\newcommand{\meccanoframefive}[8]{ %a,b,c,d,e, p,scale, x+g, g,h,i,j,k, x
\begin{tikzpicture}
\def\p{#6}
\def\xg{#8}
\pgfmathsetmacro{\ad}{#1 + #4} % a+d
\pgfmathsetmacro{\be}{#2 + #5} % b+e
\pgfmathsetmacro{\cosb}{(#1^2 + #3^2 - #2^2)/(2*#1*#3)} % (a^2 + c^2 - b^2)/(2ac)
\pgfmathsetmacro{\cosa}{(#2^2 + #3^2 - #1^2)/(2*#2*#3)} % (b^2 + c^2 - a^2)/(2bc)
\pgfmathsetmacro{\x}{max(#3,\xg)} % max(c,x+g)
\begin{scope}[scale=#7] 
 \path (0,0) ++(180:5*\p) node{$A$};
 \meccanostrip[0000cc]{\x}{1}{\p}
 \begin{scope}[rotate={acos(\cosb)}]
  \meccanostrip[cc0000]{\ad}{1}{\p} % red a+d=CB+BE
  \path(#3,0) ++(-45:5*\p) node{$C$};
  \path(\be,0) ++(0:5*\p) node{$D$};
 \end{scope}
 \begin{scope}[shift={(#3,0)},rotate=180-acos{\cosa}] % shift \c
  \meccanostrip[008800]{\be}{1}{\p} % green b+e=CA+AD
  \path(0,0) ++(-90:5*\p) node{$B$};
  \path(\ad,0) ++(0:5*\p) node{$E$};
 \end{scope}
 \meccanoframefivebelow
}

\newcommand{\meccanoframefivebelow}[6]{ % g,h,i,j,k, x
\def\x{#6}
\pgfmathsetmacro{\gj}{#1 + #4} % g+j
\pgfmathsetmacro{\hk}{#2 + #5} % h+k
\pgfmathsetmacro{\cosb}{(#1^2 + #3^2 - #2^2)/(2*#1*#3)} % (a^2 + c^2 - b^2)/(2ac)
\pgfmathsetmacro{\cosg}{(#2^2 + #3^2 - #1^2)/(2*#2*#3)} % (b^2 + c^2 - a^2)/(2bc)
 \begin{scope}[shift={(\x,0)},rotate={-acos(\cosg)}]
  \meccanostrip[cc00cc]{\gj}{1}{\p}
  \path (0,0) ++(180:5*\p) node{$G$};
  \path (#3,0) ++(45:5*\p) node{$I$};
  \path (\gj,0) ++(0:5*\p) node{$K$};
 \end{scope}
 \begin{scope}[shift={(\x+#3,0)},rotate={180+acos(\cosb)}]
  \meccanostrip[cc8800]{\hk}{1}{\p}
  \path (0,0) ++(180:5*\p) node{$H$};
  \path (\hk,0) ++(0:5*\p) node{$J$};
 \end{scope}

\end{scope} %above
\end{tikzpicture} %above
}


\begin{document}

\maketitle
\begin{abstract}
Meccano frames are groups of meccano\meccanoref strips intended to be a
base to build diverse meccano larger objects.
\end{abstract}

\section{Triangular frame}

\begin{figure}[h]
\centering
\meccanoframetriangle{5}{4}{7}{3}{1}{3pt}{0.75}
\caption{Triangular frame. With three strips we form the triangle $\triangle{ABC}$.
At least we extend one of two strips $\overline{CB}$ and $\overline{CA}$ to become
$\overline{CE}$ and $\overline{CD}$. The new vertices $D$ and $E$ are rigid as the
triangle and we'll calculate the distance between them.}
\label{fig:triangle}
\end{figure}

Figure \ref{fig:triangle} shows a triangular frame with strips with extentions. 
First we define five integer distances $a,b,c,d,e$ and calculate the cosine of $\angle{BCA}$:
\begin{align}
a \equiv \overline{CB}, \quad b \equiv \overline{CA}  \quad c &\equiv \overline{AB}
 \quad d \equiv \overline{BE}, \quad e \equiv \overline{AD} \\
\theta &\equiv \angle{BCA} \\
\cos\theta &= \frac{a^2 + b^2 - c^2}{2ab}
\end{align}
Then we apply the cosine to the triangle $\triangle{CED}$ to get the
extensions distance $\overline{DE}$:
\begin{align}
\overline{DE}^2 &= \overline{CD}^2 + \overline{CE}^2
 - 2\overline{CD} \times \overline{CE}\cos\theta \nonumber\\
 &= (a+d)^2 + (b+e)^2 - 2(a+d)(b+e)\cos\theta \nonumber\\
 &= (a+d)^2 + (b+e)^2 - (a+d)(b+e)\left(\frac{a^2 + b^2 - c^2}{ab}\right)
\end{align}
We expect at most a value of the form $\sqrt{s}/t$ where $s, t \in \bbb{Z}$
so we define the surd as:
\begin{align}
 \overline{DE} &= \frac{\sqrt{s}}{t}
 = \sqrt{(a+d)^2 + (b+e)^2 - (a+d)(b+e)\left(\frac{a^2 + b^2 - c^2}{ab}\right)}
\end{align}

\subsection{Software for distance $\sqrt{s}$}

We write a factory to build all the triangles with a given surd
$\sqrt{s}$ for a given maximum limit for the distances $a+d,b+e,c$.
We reject $t \neq 1$ and $s$ as not square-free, which includes pythagorean triangles.
Next list show all the triangles with $s = \sqrt{7}, t=1$ where $a+d,b+e,c \leq 10$:
\begin{lstlisting}
NewFrames().TriangleSurds surd=7 max=10
  1) a=1 b+e=1+2 c=1 cos=1/2
  2) a+d=1+1 b+e=1+2 c=1 cos=1/2
  3) a+d=1+2 b=1 c=1 cos=1/2
  4) a+d=1+2 b+e=1+1 c=1 cos=1/2
  5) a=2 b+e=2+1 c=2 cos=1/2
  6) a+d=2+1 b=2 c=2 cos=1/2
  7) a=3 b+e=2+2 c=2 cos=3/4 CED=pi/2
  8) a+d=3+1 b+e=2+1 c=2 cos=3/4 CDE=pi/2
  9) a+d=4+2 b+e=4+4 c=1 cos=31/32
 10) a+d=4+4 b+e=4+2 c=1 cos=31/32
 11) a=7 b+e=5+1 c=3 cos=13/14
 12) a=7 b+e=5+2 c=3 cos=13/14
\end{lstlisting}
The code is in file \texttt{github.com/heptagons/meccano/frames/frames.go}.



\begin{figure}[h]
\centering
\meccanoframetriangle{3}{2}{2}{0}{2}{3pt}{1.0}
\caption{Triangle with $\overline{DE}=\sqrt{7}$ where $a=3, b+e=2+2, c=2, cos=\frac{3}{4}, CED=\pi/2$.}
\end{figure}


\section{Distance $\sqrt{s} + h$}


\begin{figure}[h]
\centering
\begin{tikzpicture}
\begin{scope}[scale=0.4]
\begin{scope}
\draw[red,dotted] (0,0) -- node[below,black]{$\sqrt{s}$}++(7,0);
\draw[black,fill=black] (7,0) node[below]{$D$}  circle(4pt)
-- node[below]{$h$} ++(5,0) node[below right]{$G$} circle(4pt)
-- node[right]{$i$} ++(-3,6) node[above]{$H$} circle(4pt)
-- node[above left]{$f$} ++(-3,-2) node[above left]{$C$} circle(4pt)
-- node[above left]{$a$} ++(-3,-2) node[above left]{$B$} circle(4pt)
-- node[above left]{$d$} ++(-3,-2) node[below left]{$E$} circle(4pt);
\draw[blue,fill=blue] (7,0)
-- node[right]{$e$} ++(-.25,1) node[above right]{$A$} circle(4pt)
-- node[right]{$b$} ++(-.75,3)
   ++(+.75,-3)
-- node[above]{$c$} (3,2);
\draw[black](6,-3) node{$(a)$};

\end{scope}

\begin{scope}[shift={(18,0)}]
\draw[red,dotted] (0,0) -- node[below,black]{$\sqrt{s}$}++(7,0);
\draw[black,fill=black] (7,0) node[below]{$D$}  circle(4pt)
-- node[below]{$h$} ++(5,0) node[below right]{$G$} circle(4pt)
-- node[right]{$i$} ++(-3,6) node[above]{$H$} circle(4pt)
-- node[above]{$g$} ++(-9,-6) node[below left]{$E$} circle(4pt);
\draw[black](6,-3) node{$(b)$};
\end{scope}


\end{scope}
\end{tikzpicture}
\caption{The five strips intented to form an algebraic distance $\overline{EG} = \sqrt{s} + h$.}
\label{fig:alg-not-right}
\end{figure}

From figure \ref{fig:alg-not-right} $(a)$ we know $\sqrt{s}$ distance
between nodes $E$ and $D$ is produced by the three strips frame $a+d$, $b+e$ and $c$.
Using the law of cosines we calculate the angle $\theta = \angle{CED}$ in terms of $\sqrt{s}$:

\begin{align}
\cos\theta &= \frac{(a+d)^2 + (\sqrt{s})^2 - (b+e)^2}{2(a+d)\sqrt{s}} \nonumber\\
 &= \frac{((a+d)^2 + s - (b+e)^2)\sqrt{s}}{2(a+d)s} \\
 &= \frac{m\sqrt{s}}{n} \label{eq:cos1}\\
 m &= (a+d)^2 + s - (b+e)^2 \\
 n &= 2(a+d)s
\end{align}

From figure \ref{fig:alg-not-right} $(a)$ we notice two sets of points are collinear:
$\{ E,B,C,H \}$ and $\{ E,D,G \}$. Using the law of cosines we calculate the 
angle $\theta = \angle{HEG}$ in terms of distances $g,h,i$:

\begin{align}
\cos\theta &= \frac{g^2 + (\sqrt{s}+h)^2 - i^2}{2g(\sqrt{s}+h)} \nonumber\\
 &= \frac{g^2 + s + 2\sqrt{s}h + h^2 - i^2}{2g(\sqrt{s}+h)} \nonumber\\
 &= \frac{g^2 + s + h^2 - i^2+ 2\sqrt{s}h}{2g(\sqrt{s}+h)}
\end{align}

We multiply both numerator and denominator by $\sqrt{s}-h$ to eliminate the surd from denominator:
\begin{align}
\cos\theta &= \frac{(s + g^2 + h^2 - i^2)(\sqrt{s}-h) + 2\sqrt{s}h(\sqrt{s}-h)}
	{2g(\sqrt{s}+h)(\sqrt{s}-h)} \nonumber\\
 &= \frac{(s + g^2 + h^2 - i^2)(\sqrt{s}-h) + 2sh - 2\sqrt{s}h^2}
	{2g(s-h^2)} \nonumber\\ 
 &= \frac{-h(s + g^2 + h^2 - i^2 - 2s) + (s + g^2 + h^2 - i^2 - 2h^2)\sqrt{s}}
	{2g(s-h^2)} \nonumber\\ 
 &= \frac{h(s - g^2 - h^2 + i^2) + (s + g^2 - h^2 - i^2)\sqrt{s}}
	{2g(s-h^2)} \nonumber\\ 
 &= \frac{o + p\sqrt{s}}{q} \label{eq:cos2}\\
o &= h(s - g^2 - h^2 + i^2) \\
p &= s + g^2 - h^2 - i^2 \\
q &= 2g(s-h^2)
\end{align}

We compare both cosines equations \ref{eq:cos1} and \ref{eq:cos2}:
\begin{align}
\frac{m\sqrt{s}}{n} &= \frac{o + p\sqrt{s}}{q}
\end{align}
Since all variables are integers we need two conditions. First $o$ should be zero.
And second $\frac{m}{n} = \frac{p}{q}$.

For condition 1, we force $o$ to be zero:
\begin{align}
o &= 0 \nonumber\\
h(s - g^2 - h^2 + i^2) & = 0 \nonumber\\
s &= g^2 + h^2 - i^2 \label{eq:condition1}
\end{align}

For condition2, we force $m,n,p,q$ as:
\begin{align}
\frac{m}{n} &= \frac{p}{q} \nonumber\\
\frac{(a+d)^2 + s - (b+e)^2}{2(a+d)s} &= \frac{s + g^2 - h^2 - i^2}{2g(s-h^2)} \nonumber\\
\end{align}

We replace the value of $s$ of last equation RHS with the value of equation \ref{eq:condition1}
of condition 1:
\begin{align}
\frac{(a+d)^2 - (b+e)^2 + s}{(a+d)s} &= \frac{s + g^2 - h^2 - i^2}{g(s-h^2)} \nonumber\\
 &= \frac{g^2 + h^2 - i^2 + g^2 - h^2 - i^2}{g(g^2 + h^2 - i^2-h^2)} \nonumber\\
 &= \frac{2(g^2 - i^2)}{g(g^2 - i^2)} \nonumber\\
 &= \frac{2}{g} \nonumber\\
((a+d)^2 - (b+e)^2 + s)g &= 2(a+d)s \label{eq:condition2}
\end{align}



\section{Five strips}

\begin{figure}[h]
\centering
% a,b,c,d,e,p,scale, x+g, g,h,i,j,k, x,p
\meccanoframefive{5}{6}{5}{4}{3}{3pt}{0.75} {7} {3}{4}{4}{3}{2} {3}
\caption{Five strips.}
\end{figure}










\end{document}
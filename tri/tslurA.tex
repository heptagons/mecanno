\PassOptionsToPackage{svgnames}{xcolor}
\documentclass[tikz, border=1cm]{standalone}

\usepackage{tikz}
\usetikzlibrary{calc}
\usetikzlibrary{math}

\begin{document}

\newcommand\nodes{{"C","B","A"}}
\newcommand{\meccanotriangle}[6]
{
\tikzmath{
	\x = #3; \y = #4; \z = #5;
	\xx = 1/\x; \yy = 1/\y; \zz = 1/\z;
	\cosX = (\y*\y + \z*\z - \x*\x)/(2*\y*\z); \sinX = sqrt(1 - \cosX*\cosX);
	\cosY = (\z*\z + \x*\x - \y*\y)/(2*\z*\x); \sinY = sqrt(1 - \cosY*\cosY);
	\cosZ = (\x*\x + \y*\y - \z*\z)/(2*\x*\y);
	coordinate \pX, \pY, \pZ;
	\pX = (0,0); %bottom left vertice
	\pY = (\z,0); %bottom right vertice
	\pZ = (\y*\cosX, \y*\sinX); %top vertice
	\xn = \nodes[0]; \yn = \nodes[1]; \zn = \nodes[2];
}

\begin{scope}[shift={(#1,#2)},scale={0.5}]

\draw[black] (\pX)
-- (\pY) foreach \t in {0,\zz,...,1} { } node[midway,below]{$2\sqrt{6}$}
-- (\pZ) foreach \t in {0,\xx,...,1} { pic [pos=\t] {code={\fill circle[radius=0.07];}}}
-- (\pX) foreach \t in {0,\yy,...,1} { pic [pos=\t] {code={\fill circle[radius=0.07];}}}
;
\draw[gray] (\pX) node[below]{\xn};
\draw[gray] (\pY) node[below]{\yn};
\draw[gray] (\pZ) node[above]{\zn};
\end{scope}
}

\begin{tikzpicture}
\meccanotriangle{ 0}  {0}{2}{3}{2*sqrt(6)};
\meccanotriangle{ 3.2}{0}{3}{3}{2*sqrt(6)};
\meccanotriangle{ 6.4}{0}{1}{4}{2*sqrt(6)};
\meccanotriangle{ 9.6}{0}{2}{4}{2*sqrt(6)};
\meccanotriangle{12.8}{0}{3}{4}{2*sqrt(6)};
\meccanotriangle{16}{0}{4}{4}{2*sqrt(6)};
\end{tikzpicture}

\end{document}


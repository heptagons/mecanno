\documentclass{article}
\usepackage{amsmath}
\usepackage{../../meccano}
\usepackage{chngcntr}
\counterwithin{equation}{section}

\begin{document}



\section{Polygons algebraic integers}
We develop code to operate complicated numbers appearing in polygons constructions, like the number:
\begin{align}
\cos{\frac{2\pi}{17}} &=
\frac{-1+\sqrt{17}+\sqrt{34-2\sqrt{17}}+2\sqrt{17+3\sqrt{17}-\sqrt{170+38\sqrt{17}}}}{16}
\end{align}

Define $A_0$, $A_1$, $A_2$ and $A_3$ algebraic integers in the numerator of equation (1.1):
\begin{align}
A_0 &= \pm b\\
A_1 &= \pm c\sqrt{\pm d}\\
A_2 &= \pm e\sqrt{f \pm g\sqrt{\pm h}}\\
A_3 &= \pm i\sqrt{j \pm k\sqrt{\pm l} \pm m\sqrt{\pm n \pm o\sqrt{\pm p}}}
\end{align}
Define a function $F$ and apply it to the variables above $b$ to $p$:
\begin{align}
 F &= F(x,F,F,...,F)\\
   &= x\sqrt{F + F + ... + F}\\
 F_0(b) &= F(b)\\
 F_1(c,d) &= F(c,F_0(d))\\
 F_2(e,f,g,h) &= F(e,F_0(f),F_1(g,h))\\
 F_3(i,j,k,l,m,n,o,p) &= F(i,F_0(j),F_1(k,l),F_2(m,n,o,p))
\end{align}

So first equation can be expressed as:
\begin{align}
\cos{\frac{2\pi}{17}} &=
\frac{F_0(-1) + F_1(1,17) + F_2(1,34,-2,17) + F_3(2,17,3,17,-1,170,38,17)}{16}
\end{align}



\subsection{32 bits limits}

Algebraic integers \textbf{parts} are expresed as $AZ32$ golang objects.
The variable $o$ (see line 2) is the number's part outside square root.
The array $i$ (see line 3) is the number's parts sum inside square root.
Symbollicaly:
\begin{align}
a &= F(o,F(o,...),...)\\
n &= len(a.i)\\
value &= \begin{cases}
 a.o                               &n = 0\\
 a.o\sqrt{ \sum_{j=0}^{n} a.i[j] } &n > 0\\
\end{cases}
\end{align}

\begin{lstlisting}
type AZ32 struct {
	o int32
	i []*AZ32
}

type AZ32s struct { // factory
	
}

func (a *AZ32s) F0(b int32) *AZ32 {
	return &AZ32{
		o: b,
	}
}

func (a *AZ32s) F1(c, d int32) *AZ32 {
	return &AZ32 {
		o: c,
		i: []*AZ32{
			a.F0(d),
		},
	}
}

func (a *AZ32s) F2(e, f, g, h int32) *AZ32 {
	return &AZ32 {
		o: e,
		i: []*AZ32{
			a.F0(f),
			a.F1(g, h),
		},
	}
}

func (a *AZ32s) F3(i, j, k, l, m, n, o, p int32) *AZ32 {
	return &AZ32 {
		o: i,
		i: []*AZ32{
			a.F0(j),
			a.F1(k, l),
			a.F2(m, n, o, p),
		},
	}
}
\end{lstlisting}



\subsection{N32, I32, AI32, AQ32}

\setlength{\columnsep}{30pt}
\begin{multicols}{2}
\begin{lstlisting}
type N32 uint32 // range 0 - 0xffffffff

type I32 struct {
	s bool // sign: true means negative
	n N32  // positive value
}

type AI32 struct {
	o *I32  // outside radical
	i *I32  // inside radical square-free
	e *AI32 // inside radical extension
}

type AQ32 struct {
	nums []*AI32 // numerator sum
	den  N32     // denominator
}
\end{lstlisting}
In this list we define four 32 bit numbers in Golang code.\\
In line 1 we define the natural number $N32$ with a range of $0 < n \leq 2^{32} - 1$.\\
In line 3 we define the integer number $I32$, the number sign is negative if $s$ is true 
and the number value always is a positive. If $I32$ is nil, then we assume the number
is zero.\\
In line 8 we define the algebraic integer number $AI32$. The number is recursive with a value of
\begin{align}
\pm o\sqrt{\pm i \pm e.o\sqrt{\pm e.i \pm e.e.o ... }}
\end{align}
where each sign $\pm$ corresponds to its integer sign $s$ of the values of integers $o$ and $i$;
a $AI32$ with value nil corresponds to a zero.
In line 14 we define the algebraic rational number $AQ32$. In the numerator has a sum of several
integer numbers $AI32$ and in the denominator a natural number $N32$ other than zero.
\end{multicols}

\subsection{Reductions $R32$}
Reductions factory $R32$ produce irreducibles $AI32$ and $AQ32$ using a precomputed fixed list of 32-bit primes. The reduction simplify and standarize the algebraic numbers for further
operations like cloning, addition, multiplication, inversion and square root extractions with results also reduced.
\begin{lstlisting}
type Red32 struct {
	primes []N32
}

func NewRed32() *Red32 {
	value := 0xffff
    f := make([]bool, value)
    for i := 2; i <= int(math.Sqrt(float64(value))); i++ {
        if f[i] == false {
            for j := i * i; j < value; j += i {
                f[j] = true
            }
        }
    }
    primes := make([]N32, 0)
    for i := N32(2); i < N32(value); i++ {
        if f[i] == false {
            primes = append(primes, i)
        }
    }
	return &Red32{
		primes: primes,
	}
}
\end{lstlisting}
Numbers of the form $x\sqrt{y}$ are reduced with a function call named $roi(x,y)$.
Numbers of the form $x\sqrt{y + z\sqrt{...}}$ are reduced with function named $roie(x,y,z)$.
The factory produces $AI32$ numbers calling both functions as necessary to return
irreducible algebraic integers. As an example, this is the process to reduce number $A_3$:
\begin{align}
A_3 &= \pm i\sqrt{\pm j \pm k\sqrt{\pm l \pm m\sqrt{\pm n}}}\\
 &\qquad \boxed{ m_1, n_1 = roi(m, n) }\\
 &= \pm i\sqrt{\pm j \pm k\sqrt{\pm l \pm m_1\sqrt{\pm n_1}}}\\
 &\qquad \boxed{ k_1, l_1, m_2 = roie(k, l, m_1) }\\
 &= \pm i\sqrt{\pm j \pm k_1\sqrt{\pm l_1 \pm m_2\sqrt{\pm n_1}}}\\
 &\qquad \boxed{ i_1, j_1, k_2 = roie(i, j, k_1) }\\
 &= \pm i_2\sqrt{\pm j_2 \pm k_3\sqrt{\pm l_1 \pm m_2\sqrt{\pm n_1}}}
\end{align}

\subsection{Reduction $roi$}
This reduction is done for $AI32$ numbers parts \textbf{without} extension $e$. This is the case of whole part $\pm c\sqrt{\pm d}$ of $A_1$,
part $\pm g\sqrt{\pm h}$ of $A_2$ and
part $\pm m\sqrt{\pm n}$ of $A_3$.
Example of reducing $A_1 = \pm c\sqrt{\pm d}$.
From $d$ find two numbers $p$ and $d_1$, so $p$ is the product of some primes even repeated
and $d_1$ is square-free or $1$:
\begin{align}
A_1 &= \pm c\sqrt{\pm d}\\
 &\qquad \boxed{ d = p^2d_1 }\\
A_1 &= \begin{cases}
 0                    &\text{case 1: if } c = 0 \text{ or } d = 0\\
 \pm cp\sqrt{+1}      &\text{case 2: if } d_1 = +1\\
 \pm c\sqrt{\pm d}    &\text{case 3: if } p = 1\\
 \pm cp\sqrt{\pm d/p} &\text{case 4: otherwise }\\
\end{cases}
\end{align}
For case 1 and case 2 we got $A_1$ degenerated into $A_0$.
For case 3 the number remains the same because was irreducible. For case 4 the reduction
updates the values where $p > 1$ and $c_1=cp$ and $d_1 = d/p$.
\begin{lstlisting}
func (r *Red32) roi(out, inA Z) (ai *AI32, overflow bool) {
	if out == 0 || inA == 0 { // case 1
		return nil, false // zero
	}
	io := out; if out < 0 { io = -out }
	ia := inA; if inA < 0 { ia = -inA }
	if no, na, overflow := r.roiN(N(io), N(ia)); overflow {
		return nil, true
	} else { // cases 2,3,4
		return &AI32{
			o: &I32{ n:no, s:out < 0 },
			i: &I32{ n:na, s:inA < 0 },
		}, false
	}
}

func (r *Red32) roiN(out, in N) (o N32, i N32, overflow bool) {
	if out == 0 || in == 0 {
		return 0, 0, false // zero
	}
	for _, prime := range r.primes {
		p := N(prime)
		if pp := p*p; in >= pp {
			for {
				if in % pp == 0 { // reduce ok
					out *= p
					in  /= pp
					continue // look for repeated squares in reduced in
				}
				break // check next prime
			}
		} else { // in has no more factors to check
			break
		}
	}
	if out > N32_MAX || in > N32_MAX {
		return 0, 0, true // overflow
	}
	return N32(out), N32(in), false
}
\end{lstlisting}

\subsection{Reduction $roie$}
This reduction is done for $AI32$ numbers \textbf{with} extension $e$. This is the case of 
part $\pm e\sqrt{\pm f + g\sqrt{...}}$ of $A_2$,
part $\pm i\sqrt{\pm j + k\sqrt{...}}$ of $A_3$ and
part $\pm k\sqrt{\pm l + m\sqrt{...}}$ of $A_3$.
Example of reducing $A_2$. First we reduce part $\pm g\sqrt{\pm h}$ with $roi(g, h)$
and apply the four cases to A2:
\begin{align}
A_1 &= \pm g\sqrt{\pm h}\\
 &\qquad \boxed{ h = p^2h_1 }\\
A_2 &= \pm e\sqrt{f \pm gp\sqrt{\pm h_1}}\\
 A_2 &= \begin{cases}
 0                                       &case 5: \text{if } e = 0\\
 \pm e\sqrt{\pm f}                       &case 6: \text{if } g = 0 \text{ or } h = 0\\
 \pm e\sqrt{\pm f \pm gp}                &case 7: \text{if } h_1 = +1\\
 \pm e\sqrt{\pm f \pm gp\sqrt{\pm h_1}} &case 8: \text{if } p \geq 1\\
\end{cases}
\end{align}
For case 5 we have that $A_2$ is zero.
For case 6 we reduce $A_2$ with $roi(\pm e,\pm f)$.
For case 7 we reduce $A_2$ with $roi(\pm e,\pm f\pm gp)$.
\\
\\
For case 8 we reduce $A_2$ with $roia(\pm e,\pm f, \pm gp)$ where $h_1$ is irreducible.
Reduction $roie$ look for another product of primes $q$ such that at the same time 
$f = q^2f_1$ and $gp = q^2g_1$:
\begin{align}
A_2 &= \pm e\sqrt{\pm f \pm gp\sqrt{\pm h1}}\\
 &\qquad \boxed{ f = q^2f_1 }\\
 &\qquad \boxed{ gp = q^2g_1 }\\
A_2 &= \pm e_1\sqrt{\pm f_1 \pm g_1\sqrt{\pm h_1}}
\end{align}
Where $e_1 = eq$, $f_1 = f/{q^2}$, $g_1 = gp/{q^2}$.
\begin{lstlisting}
func (r *Red32) roie(out, inA, inB Z) (o, i, j Z, overflow bool) {
	if out == 0 { // case 1
		return 0, 0, 0, false // zero
	}
	io := out; if out < 0 { io = -out }
	ia := inA; if inA < 0 { ia = -inA }
	ib := inB; if inB < 0 { ib = -inB }
	if no, na, nb, overflow := r.roieN(N(io), N(ia), N(ib)); overflow {
		return 0, 0, 0, true // overflow
	} else {
		zo := Z(no); if out < 0 { zo = -zo }
		za := Z(na); if inA < 0 { za = -za }
		zb := Z(nb); if inB < 0 { zb = -zb }
		return zo, za, zb, false
	}
}

func (r *Red32) roieN(out, inA, inB N) (o, i, j N32, overflow bool) {
	if inA > 1 && inB > 1 {
		for _, prime := range r.primes {
			p := N(prime)
			pp := p*p
			for {
				if inA % pp == 0 && inB % pp == 0 {
					out *= p  // multiply by p
					inA /= pp // divide by p squared
					inB /= pp // divide by p squared
					continue
				} else {
					break
				}
			}
		}
	}
	if out > N32_MAX || inA > N32_MAX || inB > N32_MAX {
		return 0, 0, 0, true // overflow
	}
	return N32(out), N32(inA), N32(inB), false 
}
\end{lstlisting}

\subsection{B, D, H, N}

We define four numbers of increasing complexity:
\begin{align}
B &\equiv \frac{A_0}{a}\\
D &\equiv \frac{A_0 + A_1}{a}\\
H &\equiv \frac{A_0 + A_1 + A_2}{a}\\
N &\equiv \frac{A_0 + A_1 + A_2 + A_3}{a}
\end{align}


\section{functions}
Each of the radicals $r_0,...,r_3$ has a function to read their corresponding signs and integers variables:
\begin{align}
f_0 &\equiv f(\pm b)\\
f_1 &\equiv f(\pm c, d)\\
f_2 &\equiv f(\pm e, f, \pm g, h)\\
f_3 &\equiv f(\pm i, j, \pm k, l, \pm m, n)
\end{align}
Each $f_0,...f_4$ reduces the values with $\gcd$ and root simplifications.

Each of the algebraic numbers $B$, $D$, $H$ and $N$ has a function to read their radicals functions as inputs:
\begin{align}
f_B &\equiv f(f_0(...), a)\\
f_D &\equiv f(f_0(...), f_1(...), a)\\
f_H &\equiv f(f_0(...), f_1(...), f_2(...), a)\\
f_N &\equiv f(f_0(...), f_1(...), f_2(...), f_3(...), a)
\end{align}
Each $f_B,...f_N$ adds the radicals reducing once more the variables with $\gcd$ root simplifications
and now considering the denominator $a$.

\section{Examples}

\subsection{$f_B$ examples}
\begin{align}
\cos{0}             &= 1           \implies f_B(f_0(1),1)\\
\sin{\frac{\pi}{6}} &= \frac{1}{2} \implies f_B(f_0(1),2)
\end{align}

\subsection{$f_D$ examples}
\begin{align}
\sin{\frac{\pi}{4}}  &= \frac{\sqrt{2}}{2}    \implies f_D(\emptyset,f_1(1,2),2)\\
\sin{\frac{\pi}{10}} &= \frac{-1+\sqrt{5}}{4} \implies f_D(f_0(-1),f_1(1,5),4)
\end{align}

\subsection{$f_H$ examples}
\begin{align}
\sin{\frac{\pi}{5}}  &= \frac{\sqrt{10-2\sqrt{5}}}{4} \implies f_H(\emptyset,\emptyset,f_2(1,10,-2,5),4)\\
\sin{\frac{\pi}{12}} &= \frac{\sqrt{6} + \sqrt{2}}{4} \implies f_H(\emptyset,f_1(1,6),f_2(1,2,0,0),4) *\\
\sin{\frac{\pi}{12}} &= \frac{\sqrt{2 + \sqrt{3}}}{2} \implies f_H(\emptyset,\emptyset,f_2(1,2,1,3),2)\\
\cos{\frac{\pi}{15}} &= \frac{1+\sqrt{5}+\sqrt{30-6\sqrt{5}}}{8} \implies f_E(f_0(1),f_1(1,5),f_2(1,30,-6,5),8)
\end{align}

\subsection{$f_N$ examples}
\begin{align}
\cos{\frac{\pi}{16}} &= \frac{\sqrt{2+\sqrt{2+\sqrt{2}}}}{2}
  \\&\implies f_N(\emptyset,\emptyset,\emptyset,f_3(1,2,1,2,1,2),2) \nonumber \\
\cos{\frac{\pi}{24}} &= \frac{\sqrt{2+\sqrt{2+\sqrt{3}}}}{2}
  \\&\implies f_N(\emptyset,\emptyset,\emptyset,f_3(1,2,1,2,1,3),2) \nonumber \\
\cos{\frac{2\pi}{17}} &=
\frac{-1+\sqrt{17}+\sqrt{34-2\sqrt{17}}+2\sqrt{17+3\sqrt{17}-\sqrt{170+38\sqrt{17}}}}{16}
    \\&\implies f_N(f_0(-1),f_1(1,17),f_2(1,34,-2,17),f_3(2,17,3,17,-1,170,+38,17),16) \nonumber \\
\end{align}

\section{Operations with result B}

\subsection{NewB $B = B_1$}
\begin{align}
B_1 &= \frac{\pm b_1}{a_1}\\
 &\quad \textbf{ Reduce } \{ a,b \} = \{a_1/G,b_1/G\} \iff G = \gcd \{a_1,b_1\} > 1 \nonumber \\
 &= \frac{\pm b}{a}
\end{align}

\subsection{AddBB $B = B_2 + B_3$}
\begin{align}
B_2 + B_3 &= \frac{\pm b_2}{a_2} + \frac{\pm b_3}{a_3} \\
 &= \frac{\pm b_2a_3 \pm b_3a_2 }{a_2a_3} = \frac{q}{p} \\
 &\quad \textbf{ Reduce } \{ a_1,b_1 \} = \{p/G, q/G\} \iff G = \gcd \{ p,q \} > 1 \nonumber \\
 &= \frac{\pm b_1 }{ a_1 } \textbf{ Solve as NewB }
\end{align}

\subsection{MulBB $B = B_2 \times B_3$}
\begin{align}
B_2 \times B_3 &= \frac{\pm b_2}{a_2} \times \frac{\pm b_3}{a_3} \\
  &= \frac{\pm b_2b_3}{a_2a_3} = \frac{q}{p} \\
 &\quad \textbf{ Reduce } \{ a_1,b_1 \} = \{p/G, q/G\} \iff G = \gcd \{ p,q \} > 1 \nonumber \\
 &= \frac{\pm b_1 }{ a_1 } \textbf{ Solve as NewB }
\end{align}

\subsection{InvB $B = 1 / B_2$}
\begin{align}
\frac{1}{B_2} &= \frac{1}{\pm b_2 / a_2}\\
 &= \frac{\pm a_2}{b_2} = \frac{q}{p} \\
 &\quad \textbf{ Reduce } \{ a_1,b_1 \} = \{p/G, q/G\} \iff G = \gcd \{ p,q \} > 1 \nonumber \\
 &= \frac{\pm b_1 }{ a_1 } \textbf{ Solve as NewB }
\end{align}

\section{Operations with result D}

\subsection{NewD $D = D_1$}
\begin{align}
D_1 &= \frac{\pm b_1 \pm c_1\sqrt{d_1}}{a_1}\\
 &\quad \textbf{ Reduce } \{p,q,r\} = \{a_1/G,b_1/G,c_1/G\} \iff G = \gcd \{a_1,b_1,c_1\} > 1 \nonumber \\
 &= \frac{\pm q \pm r\sqrt{d_1}}{p} \\
 &\quad \textbf{ Reduce } \{ d \} = s^2d_1 \iff s>1 \nonumber \\
 &= \frac{\pm q \pm rs\sqrt{d}}{p}\\
 &\quad \textbf{ Reduce } \{a,b,c\} = \{p/G,q/G,rs/G\} \iff G = \gcd \{p,q,rs\} \nonumber \\
 &= \frac{\pm b \pm c\sqrt{d}}{a}
\end{align}

\subsection{SqrtB $D = \sqrt{B_2}$}
\begin{align}
\sqrt{B_2} &= \sqrt{\frac{\pm b_2}{a_2}}\\
 &= \frac{\sqrt{a_2b_2}}{a_2}\\
 &\quad \textbf{ Set } \{ a_1, b_1, c_1, d_1\} = \{ a_2, 0, 1, a_2b_2\} \nonumber \\
 &= \frac{\pm b_1 \pm c_1\sqrt{d_1}}{a_1} \textbf{ Solve as NewD }
\end{align}

\subsection{InvD $D = 1 / D_2$}
\begin{align*}
1 / D_2 &= \frac{a_2}{\pm b_2 \pm c_2\sqrt{d_2}} \\
 &= \frac{\pm a_2b_2 \mp a_2c_2\sqrt{d_2}}{b_2^2 - c_2^2d_2}\\
 &\quad \textbf{ Set } \{ a_1, b_1, c_1, d_1\} = \{ b_2^2 - c_2^2d_2, \pm a_2b_2, \mp a_2c_2, d_2\} \nonumber\\
 &= \frac{\pm b_1 \pm c_1\sqrt{d_1}}{a_1} \textbf{ Solve as NewD } \nonumber
\end{align*}


%\subsection{$B_1 + C_2 = D_3$}
%\begin{align*}
%B_1 + C_2 &= \frac{\pm a_1}{b_1} + \frac{\pm a_2\sqrt{c_2}}{b_2}\\
% &= \frac{\pm a_2b_1\sqrt{c_2} \pm a_1b_2 }{b_1b_2}\\
% &= \frac{\pm a_3\sqrt{c_2} \pm d_3 }{b_3} & (\pm a_3, b_3, \pm d_3) = gcd{\pm a_2b_1}{b_1b_2}{\pm a_1b_2}
%\end{align*}

%\subsection{$B_1 + D_2 = D_3$}
%\begin{align*}
%B_1 + D_2 &= \frac{\pm a_1}{b_1} + \frac{\pm a_2\sqrt{c_2} \pm d_2}{b_2}\\
% &= \frac{\pm a_2b_1\sqrt{c_2} \pm a_1b_2 \pm d_2b_1 }{b_1b_2}\\
% &= \frac{\pm a_3\sqrt{c_2} \pm d_3 }{b_3} & (\pm a_3, b_3, \pm d_3) = gcd(\pm a_2b_1,b_1b_2,\pm a_1b_2 \pm d_2b_1)
%\end{align*}

%\subsection{$B_1 + F_2 = F_3$} %7
%\begin{align*}
%B_1 + F_2 &= \frac{\pm a_1}{b_1} + \frac{\pm a_2\sqrt{c_2 \pm e_2\sqrt{f_2}} \pm d_2}{b_2}\\
% &= \frac{\pm a_2b1\sqrt{c_2 \pm e_2\sqrt{f_2}} \pm a_1b_2 \pm d_2b_1}{b_1b_2}\\
% &= \frac{\pm a_3\sqrt{c_2 \pm e_2\sqrt{f_2}} \pm d_3}{b_3} & (\pm a_3, b_3, \pm d_3)=gcd(\pm a_2b1,b_1b_2,\pm a_1b_2 \pm %d_2b_1)
%\end{align*}



\section{Operations with result $H$}

\subsection{$D_1 + D_2 \mapsto H$ <<<<<}
\begin{align}
D_1 + D_2 &= \frac{\pm b_1 \pm c_1\sqrt{d_1}}{a_1} + \frac{\pm b_2 \pm c_2\sqrt{d_2}}{a_2}\\
 &= \frac{(\pm a_2b_1 \pm a_1b_2) \pm a_2c_1\sqrt{d_1} \pm a_1c_2\sqrt{d_2}}{a_1a_2}\\
 &= \frac{\pm q \pm r\sqrt{d_1} \pm s\sqrt{d_2}}{p} \\
 &\quad \textbf{ where } \{ p,q,r,s \} = \gcd \{ a_1a_2, (\pm a_2b_1 \pm a_1b_2), \pm a_2c_1, \pm a_1c_2 \} \nonumber \\
 &= \frac{\pm q \pm \sqrt{r^2d_1 + s^2d_2 \pm 2rs\sqrt{d_1d_2}}}{p} \\
 &= \frac{\pm q \pm \sqrt{t \pm 2rsu\sqrt{h}}}{p} \\
 &\quad \textbf{ where } \{ t \} = r^2d_1 + s^2d_2 \textbf{ and } \{ u^2h \} = d_1d_2 \nonumber \\
 &= \frac{\pm q \pm v\sqrt{f \pm g\sqrt{h}}}{p} \\
 &\quad \textbf{ where } \{ v^2f \} = t \textbf{ and } \{ v^2g \} = 2rsu \nonumber \\
 &= \frac{\pm d \pm e\sqrt{f \pm g\sqrt{h}}}{a} \\
 &\quad \textbf{ where } \{a,d,e\} = \gcd\{ p, \pm q, \pm qv \} \nonumber \\
\end{align}


\subsection{$\sqrt{C_1} = F_2$}
\begin{align*}
\sqrt{C_1} &= \sqrt{\frac{a_1\sqrt{c_1}}{b_1}}\\
 &= \frac{\sqrt{a_1b_1\sqrt{c_1}}}{b_1}     \\
 &= \frac{m\sqrt{e_2\sqrt{c_1}}}{b_1}       & a_1b_1 = m^2e_2\\
 &= \frac{a_2\sqrt{e_2\sqrt{c_1}}}{b_2}     & (a_2,b_2) = gcd(m,b_1)
\end{align*}

\subsection{$C_1 + D_2 = F_3$}
\begin{align*}
C_1 + D_2 &= \frac{\pm a_1\sqrt{c_1}}{b_1} + \frac{\pm a_2\sqrt{c_2} \pm d_2 }{b_2}\\
 &= \frac{\pm a_1b_2\sqrt{c_1} \pm a_2b_1\sqrt{c_2} \pm d_2b_1}{b_1b_2} \\
 &= \frac{\pm m\sqrt{c_1} \pm n\sqrt{c_2} \pm p}{o} 
 & (\pm m, \pm n, \pm p, o) = gcd(\pm a_1b_2, \pm a_2b_1, \pm d_2b_1, b_1b_2)\\
 &= \frac{\sqrt{m^2c_1 + n^2c_2 \pm 2mn\sqrt{c1c_2}} \pm p }{o} \\
 &= \frac{\sqrt{q \pm 2mnr\sqrt{f_3}} \pm p}{o}     & q = m^2c_1 + n^2c_2, c_1c_2 = r^2f_3\\
 &= \frac{s\sqrt{c_3 \pm e_3\sqrt{f_3}} \pm p}{o}   & q = s^2c_3, 2mnr = s^2e_3 \\
 &= \frac{a_3\sqrt{c_3 \pm e_3\sqrt{f_3}} \pm d_3}{b_3}
    & (a_3, b_3, \pm d_3) = gcd(s, \pm p, o)
\end{align*}

\subsection{$1 / D_1 = D_2$}
\begin{align*}
1 / D_1 &= \frac{b_1}{\pm a_1\sqrt{c_1} \pm d_1 }\\
 &= \frac{\pm a_1b_1\sqrt{c_1} \mp b_1d_1 }{a_1^2c_1 - d_1^2}\\
 &= \frac{a_2\sqrt{c_1} \pm d_2}{b_2}
    & (a_2, b_2, d_2) = gcd(\pm a_1b_1, \mp b_1d_1, a_1^2c_1 - d_1^2)
\end{align*}

\subsection{$\sqrt{D_1} = F_2$ editing...}
\begin{align*}
\sqrt{D_1} &= \sqrt{\frac{\pm a_1\sqrt{c_1} \pm d_1 }{b_1}}\\
 &= \frac{\sqrt{\pm b_1d_1 \pm a_1b_1\sqrt{f_2}}}{b_1}
    & f_2 = c_1\\
 &= \frac{m\sqrt{c_2 \pm e_2\sqrt{f_2}}}{b_1}
    & \pm b_1d_1 = m^2c_2, \pm a_1b_1 = m^2e_2\\
 &= \frac{a_2\sqrt{c_2 \pm e_2\sqrt{f_2}}}{b_2}
    & (a_2, b_2) = gcd(m, b_1)
\end{align*}

\subsection{$D_1 + D_2 = F_3$}
\begin{align*}
D_1 + D_2 &= \frac{\pm a_1\sqrt{c_1} \pm d_1 }{b_1} + \frac{\pm a_2\sqrt{c_2} \pm d_2 }{b_2}\\
 &= \frac{\pm a_1b_2\sqrt{c_1} \pm a_2b_1\sqrt{c_2} \pm d_1b_2 \pm d_2b_1}{b_1b_2} \\
 &= \frac{\pm m\sqrt{c_1} \pm n\sqrt{c_2} \pm p }{o}
     & (\pm m, \pm n, \pm p, o) = gcd(\pm a_1b_2, \pm a_2b_1, \pm d_1b_2 \pm d_2b_1, b_1b_2)\\
 &= \frac{\sqrt{m^2c_1 + n^2c_2 \pm 2mn\sqrt{c1c_2}} \pm p }{o} \\
 &= \frac{\sqrt{q \pm 2mnr\sqrt{f_3}} \pm p}{o}     & q = m^2c_1 + n^2c_2, c_1c_2 = r^2f_3\\
 &= \frac{s\sqrt{c_3 \pm e_3\sqrt{f_3}} \pm p}{o}   & q = s^2c_3, 2mnr = s^2e_3 \\
 &= \frac{a_3\sqrt{c_3 \pm e_3\sqrt{f_3}} \pm d_3}{b_3}
    & (a_3, b_3, \pm d_3) = gcd(s, \pm p, o)
\end{align*}

\subsection{$D_1 \times D_2 = F_3$}
\begin{align*}
D_1 \times D_2 &= \frac{\pm a_1\sqrt{c_1} \pm d_1 }{b_1} \times \frac{\pm a_2\sqrt{c_2} \pm d_2 }{b_2}\\
 &= \frac{\pm a_1a_2\sqrt{c_1c_2} \pm a_1d_2\sqrt{c1} \pm a_2d_1\sqrt{c_2} \pm d_1d_2}{b_1b_2}\\
\end{align*}

\subsection{MulDD $D_1 \times D_2 \mapsto H$ ???}
\begin{align*}
D_1 \times D_2 &= \frac{\pm b_1 \pm c_1\sqrt{d_1}}{a_1} \times \frac{\pm b_2 \pm c_2\sqrt{d_2}}{a_2}\\
 &= \frac{\pm b_1b_2 \pm b_1c_2\sqrt{d_2} \pm b_2c_1\sqrt{d_1} \pm c_1c_2\sqrt{d_1d_2} }{a_1a_2}\\
 &= \frac{\pm a_1a_2m\sqrt{c_3}}{b_1b_2} & c_1c_2 = m^2c_3\\
 &= \frac{\pm a_3\sqrt{c_3}}{b_3} & (\pm a_3,b_3) = gcd(\pm a_1a_2m,b_1b_2)
\end{align*}






















\end{document}
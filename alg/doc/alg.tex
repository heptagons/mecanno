\documentclass{article}
\usepackage{amsmath}
\usepackage{../../meccano}
\usepackage{chngcntr}
\counterwithin{equation}{section}

\begin{document}

\section{32 bits algebraic integers}

Let $A_0$, $A_1$, $A_2$ and $A_3$ algebraic integers with levels 0, 1, 2 and 3:
\begin{align}
A_0 &= \pm b\\
A_1 &= \pm c\sqrt{\pm d}\\
A_2 &= \pm e\sqrt{f \pm g\sqrt{\pm h}}\\
A_3 &= \pm i\sqrt{j \pm k\sqrt{l \pm m\sqrt{\pm n}}}
\end{align}

We will use fourteen different 32-bit natural numbers, where $a$ goes in the denominators and $b,...n$ 
in the numerators.
\begin{align}
1 &\leq a       \leq 2^{32} - 1\\
0 &\leq b,c,d,e,f,g,h,i,j,k,l,m,n \leq 2^{32} - 1
\end{align}
The signs are managed appart as extra boolean variables and there is one for each of the seven variables
$b$, $c$, $e$, $g$, $i$, $k$ and $m$.

\subsection{N32, I32 and AI32}

\setlength{\columnsep}{30pt}
\begin{multicols}{2}
\begin{lstlisting}
type N32 uint32 // range 0 - 0xffffffff

type I32 struct {
  s bool // sign: true means negative
  n N32  // positive value
}

type AI32 struct {
  o *I32  // outside radical
  i *I32  // inside radical square-free
  e *AI32 // inside radical extension
}
\end{lstlisting}
In this list we define three 32 bit numbers in Golang code.\\
In line 1 we define the natural number $N32$ with a range of $0 < n \leq 2^{32} - 1$.\\
In line 3 we define the integer number $I32$, the number sign is negative if $s$ is true 
and the number value always is a positive. If $I32$ is nil, then we assume the number
is zero.\\
In line 8 we define the algebraci integer number $AI32$. The number is recursive with a value of
\begin{align}
\pm o\sqrt{\pm i \pm e.o\sqrt{\pm e.i \pm e.e.o ... }}
\end{align}
where each sign $\pm$ corresponds to its integer sign $s$ of the values of integers $o$ and $i$.
\end{multicols}

\subsection{Reductions}
Reductions simplify and standarize the numbers representations. Are applied to the 
inputs and outputs numbers in operations like
copy, addition, multiplication, inversion and square roots extraction.
Any complicated abstract integer is reduced from the inside to the outside.
\begin{align}
A_3 &= \pm i\sqrt{\pm j \pm k\sqrt{\pm l \pm m\sqrt{\pm n}}}\\
 &\qquad m_1, n_1 = reduceOI(m, n) \nonumber\\
 &= \pm i\sqrt{\pm j \pm k\sqrt{\pm l \pm m_1\sqrt{\pm n_1}}}\\
 &\qquad k_1, l_1, m_2 = reduceOII(k, l, m_1) \nonumber\\
 &= \pm i\sqrt{\pm j \pm k_1\sqrt{\pm l_1 \pm m_2\sqrt{\pm n_1}}}\\
 &\qquad i_1, j_1, k_2 = reduceOII(i, j, k_1) \nonumber\\
 &= \pm i_2\sqrt{\pm j_2 \pm k_3\sqrt{\pm l_1 \pm m_2\sqrt{\pm n_1}}}
\end{align}

\subsection{Reduction $roi$}
This reduction is done for $AI32$ numbers \textbf{without} extension $e$. This is the case of
part $\pm c\sqrt{\pm d}$ of $A_1$,
part $\pm g\sqrt{\pm h}$ of $A_2$ and
part $\pm m\sqrt{\pm n}$ of $A_3$.
Example of reducing $A_1$:
\begin{align}
A_1 &= \pm c\sqrt{\pm d}\\
d   &= p^2d_1 \qquad \textbf{ From $d$ find $p,d_1$ where $d_1$ is square-free or $1$} \\
roi(c,d) &= \begin{cases}
 0                    &\text{case 1: if } c = 0 \text{ or } d = 0\\
 \pm cp               &\text{case 2: if } d_1 = +1\\
 \pm c\sqrt{\pm d}    &\text{case 3: if } p = 1\\
 \pm cp\sqrt{\pm d_1} &\text{case 4: otherwise }\\
\end{cases}
\end{align}
For cases $1$ and $2$ we got $A_1$ degenerated into $A_0$.
For case $3$ values remain the same because the number was irreducible.
For case $4$ we got reduced $A_1$ with new values $c_1=cp$ and $d_1$.

\subsection{Reduction $roie$}
This reduction is done for $AI32$ numbers \textbf{with} extension $e$. This is the case of 
part $\pm e\sqrt{\pm f + g\sqrt{...}}$ of $A_2$,
part $\pm i\sqrt{\pm j + k\sqrt{...}}$ of $A_3$ and
part $\pm k\sqrt{\pm l + m\sqrt{...}}$ of $A_3$.
Example of reducing $A_2$. First we reduce part $\pm g\sqrt{\pm h}$ with reduceOII 
function:
\begin{align}
A_1 &= \pm g\sqrt{\pm h}\\
roi(g,h) &= \begin{cases}
 \pm g_1               &\text{reduceOII cases 1 or 2}\\
 \pm g_2\sqrt{\pm h_2} &\text{reduceOII cases 3 or 4}\\
\end{cases}\\
A_2 &= \begin{cases}
 0                                       &case 1: \text{if } e = 0\\
 \pm e\sqrt{\pm f \pm g_1}               &case 2: \text{if } h_2 = +1\\
 \pm e\sqrt{\pm f \pm g_2\sqrt{\pm h_2}} &case 3: \text{otherwise }
\end{cases}
\end{align}
For case $1$ we have that $A_2$ degenerated into a $A_0$ so we finish.
For case $2$ we have that $A_2$ degenerated into a $A_1$ with values
$\pm e$ and $\pm f+g_1$ and is then reduced with function $reduceOI$.


and $3$ we have that $A_2$ degenerated into $A_1$, so we proceed to go
to reduce further this new $A_1$ as in previous section. For cases $4$ and $5$ we rewrite
the $A_2$ with reduced values $g_1$ and square-free $h_1$:
\begin{align}\\
A_2 &= \pm e\sqrt{\pm f \pm g_1\sqrt{\pm h_1}}\\
g_1 &= r^2g_2 \qquad \textbf{From $g_1$ found $r,g_2$ where $r$ matches with next equation's}\\
f   &= r^2f_1 \qquad \textbf{From $f$ found $r,f_1$ where $r$ matches with previous equation's}\\
A_2 &= \begin{cases}
 \pm e\sqrt{\pm f \pm g_1\sqrt{\pm h_1}} &case 6: \text{if } r = 1 \text{ nothing changed}\\
 \pm er\sqrt{\pm f_1 \pm g_2\sqrt{\pm h_1}}  &case 7: \text{otherwise }
\end{cases}
\end{align}

\subsection{B, D, H, N}

We define four numbers of increasing complexity:
\begin{align}
B &\equiv \frac{A_0}{a}\\
D &\equiv \frac{A_0 + A_1}{a}\\
H &\equiv \frac{A_0 + A_1 + A_2}{a}\\
N &\equiv \frac{A_0 + A_1 + A_2 + A_3}{a}
\end{align}


\section{functions}
Each of the radicals $r_0,...,r_3$ has a function to read their corresponding signs and integers variables:
\begin{align}
f_0 &\equiv f(\pm b)\\
f_1 &\equiv f(\pm c, d)\\
f_2 &\equiv f(\pm e, f, \pm g, h)\\
f_3 &\equiv f(\pm i, j, \pm k, l, \pm m, n)
\end{align}
Each $f_0,...f_4$ reduces the values with $\gcd$ and root simplifications.

Each of the algebraic numbers $B$, $D$, $H$ and $N$ has a function to read their radicals functions as inputs:
\begin{align}
f_B &\equiv f(f_0(...), a)\\
f_D &\equiv f(f_0(...), f_1(...), a)\\
f_H &\equiv f(f_0(...), f_1(...), f_2(...), a)\\
f_N &\equiv f(f_0(...), f_1(...), f_2(...), f_3(...), a)
\end{align}
Each $f_B,...f_N$ adds the radicals reducing once more the variables with $\gcd$ root simplifications
and now considering the denominator $a$.

\section{Examples}

\subsection{$f_B$ examples}
\begin{align}
\cos{0}             &= 1           \implies f_B(f_0(1),1)\\
\sin{\frac{\pi}{6}} &= \frac{1}{2} \implies f_B(f_0(1),2)
\end{align}

\subsection{$f_D$ examples}
\begin{align}
\sin{\frac{\pi}{4}}  &= \frac{\sqrt{2}}{2}    \implies f_D(\emptyset,f_1(1,2),2)\\
\sin{\frac{\pi}{10}} &= \frac{-1+\sqrt{5}}{4} \implies f_D(f_0(-1),f_1(1,5),4)
\end{align}

\subsection{$f_H$ examples}
\begin{align}
\sin{\frac{\pi}{5}}  &= \frac{\sqrt{10-2\sqrt{5}}}{4} \implies f_H(\emptyset,\emptyset,f_2(1,10,-2,5),4)\\
\sin{\frac{\pi}{12}} &= \frac{\sqrt{6} + \sqrt{2}}{4} \implies f_H(\emptyset,f_1(1,6),f_2(1,2,0,0),4) *\\
\sin{\frac{\pi}{12}} &= \frac{\sqrt{2 + \sqrt{3}}}{2} \implies f_H(\emptyset,\emptyset,f_2(1,2,1,3),2)\\
\cos{\frac{\pi}{15}} &= \frac{1+\sqrt{5}+\sqrt{30-6\sqrt{5}}}{8} \implies f_E(f_0(1),f_1(1,5),f_2(1,30,-6,5),8)
\end{align}

\subsection{$f_N$ examples}
\begin{align}
\cos{\frac{\pi}{16}} &= \frac{\sqrt{2+\sqrt{2+\sqrt{2}}}}{2}
  \\&\implies f_N(\emptyset,\emptyset,\emptyset,f_3(1,2,1,2,1,2),2) \nonumber \\
\cos{\frac{\pi}{24}} &= \frac{\sqrt{2+\sqrt{2+\sqrt{3}}}}{2}
  \\&\implies f_N(\emptyset,\emptyset,\emptyset,f_3(1,2,1,2,1,3),2) \nonumber \\
\cos{\frac{2\pi}{17}} &=
\frac{-1+\sqrt{17}+\sqrt{34-2\sqrt{17}}+2\sqrt{17+3\sqrt{17}-\sqrt{170+38\sqrt{17}}}}{16}
    \\&\implies f_N(f_0(-1),f_1(1,17),f_2(1,34,-2,17),f_3(2,17,3,17,-1,170,+38,17),16) \nonumber \\
\end{align}

\section{Operations with result B}

\subsection{NewB $B = B_1$}
\begin{align}
B_1 &= \frac{\pm b_1}{a_1}\\
 &\quad \textbf{ Reduce } \{ a,b \} = \{a_1/G,b_1/G\} \iff G = \gcd \{a_1,b_1\} > 1 \nonumber \\
 &= \frac{\pm b}{a}
\end{align}

\subsection{AddBB $B = B_2 + B_3$}
\begin{align}
B_2 + B_3 &= \frac{\pm b_2}{a_2} + \frac{\pm b_3}{a_3} \\
 &= \frac{\pm b_2a_3 \pm b_3a_2 }{a_2a_3} = \frac{q}{p} \\
 &\quad \textbf{ Reduce } \{ a_1,b_1 \} = \{p/G, q/G\} \iff G = \gcd \{ p,q \} > 1 \nonumber \\
 &= \frac{\pm b_1 }{ a_1 } \textbf{ Solve as NewB }
\end{align}

\subsection{MulBB $B = B_2 \times B_3$}
\begin{align}
B_2 \times B_3 &= \frac{\pm b_2}{a_2} \times \frac{\pm b_3}{a_3} \\
  &= \frac{\pm b_2b_3}{a_2a_3} = \frac{q}{p} \\
 &\quad \textbf{ Reduce } \{ a_1,b_1 \} = \{p/G, q/G\} \iff G = \gcd \{ p,q \} > 1 \nonumber \\
 &= \frac{\pm b_1 }{ a_1 } \textbf{ Solve as NewB }
\end{align}

\subsection{InvB $B = 1 / B_2$}
\begin{align}
\frac{1}{B_2} &= \frac{1}{\pm b_2 / a_2}\\
 &= \frac{\pm a_2}{b_2} = \frac{q}{p} \\
 &\quad \textbf{ Reduce } \{ a_1,b_1 \} = \{p/G, q/G\} \iff G = \gcd \{ p,q \} > 1 \nonumber \\
 &= \frac{\pm b_1 }{ a_1 } \textbf{ Solve as NewB }
\end{align}

\section{Operations with result D}

\subsection{NewD $D = D_1$}
\begin{align}
D_1 &= \frac{\pm b_1 \pm c_1\sqrt{d_1}}{a_1}\\
 &\quad \textbf{ Reduce } \{p,q,r\} = \{a_1/G,b_1/G,c_1/G\} \iff G = \gcd \{a_1,b_1,c_1\} > 1 \nonumber \\
 &= \frac{\pm q \pm r\sqrt{d_1}}{p} \\
 &\quad \textbf{ Reduce } \{ d \} = s^2d_1 \iff s>1 \nonumber \\
 &= \frac{\pm q \pm rs\sqrt{d}}{p}\\
 &\quad \textbf{ Reduce } \{a,b,c\} = \{p/G,q/G,rs/G\} \iff G = \gcd \{p,q,rs\} \nonumber \\
 &= \frac{\pm b \pm c\sqrt{d}}{a}
\end{align}

\subsection{SqrtB $D = \sqrt{B_2}$}
\begin{align}
\sqrt{B_2} &= \sqrt{\frac{\pm b_2}{a_2}}\\
 &= \frac{\sqrt{a_2b_2}}{a_2}\\
 &\quad \textbf{ Set } \{ a_1, b_1, c_1, d_1\} = \{ a_2, 0, 1, a_2b_2\} \nonumber \\
 &= \frac{\pm b_1 \pm c_1\sqrt{d_1}}{a_1} \textbf{ Solve as NewD }
\end{align}

\subsection{InvD $D = 1 / D_2$}
\begin{align*}
1 / D_2 &= \frac{a_2}{\pm b_2 \pm c_2\sqrt{d_2}} \\
 &= \frac{\pm a_2b_2 \mp a_2c_2\sqrt{d_2}}{b_2^2 - c_2^2d_2}\\
 &\quad \textbf{ Set } \{ a_1, b_1, c_1, d_1\} = \{ b_2^2 - c_2^2d_2, \pm a_2b_2, \mp a_2c_2, d_2\} \nonumber\\
 &= \frac{\pm b_1 \pm c_1\sqrt{d_1}}{a_1} \textbf{ Solve as NewD } \nonumber
\end{align*}


%\subsection{$B_1 + C_2 = D_3$}
%\begin{align*}
%B_1 + C_2 &= \frac{\pm a_1}{b_1} + \frac{\pm a_2\sqrt{c_2}}{b_2}\\
% &= \frac{\pm a_2b_1\sqrt{c_2} \pm a_1b_2 }{b_1b_2}\\
% &= \frac{\pm a_3\sqrt{c_2} \pm d_3 }{b_3} & (\pm a_3, b_3, \pm d_3) = gcd{\pm a_2b_1}{b_1b_2}{\pm a_1b_2}
%\end{align*}

%\subsection{$B_1 + D_2 = D_3$}
%\begin{align*}
%B_1 + D_2 &= \frac{\pm a_1}{b_1} + \frac{\pm a_2\sqrt{c_2} \pm d_2}{b_2}\\
% &= \frac{\pm a_2b_1\sqrt{c_2} \pm a_1b_2 \pm d_2b_1 }{b_1b_2}\\
% &= \frac{\pm a_3\sqrt{c_2} \pm d_3 }{b_3} & (\pm a_3, b_3, \pm d_3) = gcd(\pm a_2b_1,b_1b_2,\pm a_1b_2 \pm d_2b_1)
%\end{align*}

%\subsection{$B_1 + F_2 = F_3$} %7
%\begin{align*}
%B_1 + F_2 &= \frac{\pm a_1}{b_1} + \frac{\pm a_2\sqrt{c_2 \pm e_2\sqrt{f_2}} \pm d_2}{b_2}\\
% &= \frac{\pm a_2b1\sqrt{c_2 \pm e_2\sqrt{f_2}} \pm a_1b_2 \pm d_2b_1}{b_1b_2}\\
% &= \frac{\pm a_3\sqrt{c_2 \pm e_2\sqrt{f_2}} \pm d_3}{b_3} & (\pm a_3, b_3, \pm d_3)=gcd(\pm a_2b1,b_1b_2,\pm a_1b_2 \pm %d_2b_1)
%\end{align*}



\section{Operations with result $H$}

\subsection{$D_1 + D_2 \mapsto H$ <<<<<}
\begin{align}
D_1 + D_2 &= \frac{\pm b_1 \pm c_1\sqrt{d_1}}{a_1} + \frac{\pm b_2 \pm c_2\sqrt{d_2}}{a_2}\\
 &= \frac{(\pm a_2b_1 \pm a_1b_2) \pm a_2c_1\sqrt{d_1} \pm a_1c_2\sqrt{d_2}}{a_1a_2}\\
 &= \frac{\pm q \pm r\sqrt{d_1} \pm s\sqrt{d_2}}{p} \\
 &\quad \textbf{ where } \{ p,q,r,s \} = \gcd \{ a_1a_2, (\pm a_2b_1 \pm a_1b_2), \pm a_2c_1, \pm a_1c_2 \} \nonumber \\
 &= \frac{\pm q \pm \sqrt{r^2d_1 + s^2d_2 \pm 2rs\sqrt{d_1d_2}}}{p} \\
 &= \frac{\pm q \pm \sqrt{t \pm 2rsu\sqrt{h}}}{p} \\
 &\quad \textbf{ where } \{ t \} = r^2d_1 + s^2d_2 \textbf{ and } \{ u^2h \} = d_1d_2 \nonumber \\
 &= \frac{\pm q \pm v\sqrt{f \pm g\sqrt{h}}}{p} \\
 &\quad \textbf{ where } \{ v^2f \} = t \textbf{ and } \{ v^2g \} = 2rsu \nonumber \\
 &= \frac{\pm d \pm e\sqrt{f \pm g\sqrt{h}}}{a} \\
 &\quad \textbf{ where } \{a,d,e\} = \gcd\{ p, \pm q, \pm qv \} \nonumber \\
\end{align}


\subsection{$\sqrt{C_1} = F_2$}
\begin{align*}
\sqrt{C_1} &= \sqrt{\frac{a_1\sqrt{c_1}}{b_1}}\\
 &= \frac{\sqrt{a_1b_1\sqrt{c_1}}}{b_1}     \\
 &= \frac{m\sqrt{e_2\sqrt{c_1}}}{b_1}       & a_1b_1 = m^2e_2\\
 &= \frac{a_2\sqrt{e_2\sqrt{c_1}}}{b_2}     & (a_2,b_2) = gcd(m,b_1)
\end{align*}

\subsection{$C_1 + D_2 = F_3$}
\begin{align*}
C_1 + D_2 &= \frac{\pm a_1\sqrt{c_1}}{b_1} + \frac{\pm a_2\sqrt{c_2} \pm d_2 }{b_2}\\
 &= \frac{\pm a_1b_2\sqrt{c_1} \pm a_2b_1\sqrt{c_2} \pm d_2b_1}{b_1b_2} \\
 &= \frac{\pm m\sqrt{c_1} \pm n\sqrt{c_2} \pm p}{o} 
 & (\pm m, \pm n, \pm p, o) = gcd(\pm a_1b_2, \pm a_2b_1, \pm d_2b_1, b_1b_2)\\
 &= \frac{\sqrt{m^2c_1 + n^2c_2 \pm 2mn\sqrt{c1c_2}} \pm p }{o} \\
 &= \frac{\sqrt{q \pm 2mnr\sqrt{f_3}} \pm p}{o}     & q = m^2c_1 + n^2c_2, c_1c_2 = r^2f_3\\
 &= \frac{s\sqrt{c_3 \pm e_3\sqrt{f_3}} \pm p}{o}   & q = s^2c_3, 2mnr = s^2e_3 \\
 &= \frac{a_3\sqrt{c_3 \pm e_3\sqrt{f_3}} \pm d_3}{b_3}
    & (a_3, b_3, \pm d_3) = gcd(s, \pm p, o)
\end{align*}

\subsection{$1 / D_1 = D_2$}
\begin{align*}
1 / D_1 &= \frac{b_1}{\pm a_1\sqrt{c_1} \pm d_1 }\\
 &= \frac{\pm a_1b_1\sqrt{c_1} \mp b_1d_1 }{a_1^2c_1 - d_1^2}\\
 &= \frac{a_2\sqrt{c_1} \pm d_2}{b_2}
    & (a_2, b_2, d_2) = gcd(\pm a_1b_1, \mp b_1d_1, a_1^2c_1 - d_1^2)
\end{align*}

\subsection{$\sqrt{D_1} = F_2$ editing...}
\begin{align*}
\sqrt{D_1} &= \sqrt{\frac{\pm a_1\sqrt{c_1} \pm d_1 }{b_1}}\\
 &= \frac{\sqrt{\pm b_1d_1 \pm a_1b_1\sqrt{f_2}}}{b_1}
    & f_2 = c_1\\
 &= \frac{m\sqrt{c_2 \pm e_2\sqrt{f_2}}}{b_1}
    & \pm b_1d_1 = m^2c_2, \pm a_1b_1 = m^2e_2\\
 &= \frac{a_2\sqrt{c_2 \pm e_2\sqrt{f_2}}}{b_2}
    & (a_2, b_2) = gcd(m, b_1)
\end{align*}

\subsection{$D_1 + D_2 = F_3$}
\begin{align*}
D_1 + D_2 &= \frac{\pm a_1\sqrt{c_1} \pm d_1 }{b_1} + \frac{\pm a_2\sqrt{c_2} \pm d_2 }{b_2}\\
 &= \frac{\pm a_1b_2\sqrt{c_1} \pm a_2b_1\sqrt{c_2} \pm d_1b_2 \pm d_2b_1}{b_1b_2} \\
 &= \frac{\pm m\sqrt{c_1} \pm n\sqrt{c_2} \pm p }{o}
     & (\pm m, \pm n, \pm p, o) = gcd(\pm a_1b_2, \pm a_2b_1, \pm d_1b_2 \pm d_2b_1, b_1b_2)\\
 &= \frac{\sqrt{m^2c_1 + n^2c_2 \pm 2mn\sqrt{c1c_2}} \pm p }{o} \\
 &= \frac{\sqrt{q \pm 2mnr\sqrt{f_3}} \pm p}{o}     & q = m^2c_1 + n^2c_2, c_1c_2 = r^2f_3\\
 &= \frac{s\sqrt{c_3 \pm e_3\sqrt{f_3}} \pm p}{o}   & q = s^2c_3, 2mnr = s^2e_3 \\
 &= \frac{a_3\sqrt{c_3 \pm e_3\sqrt{f_3}} \pm d_3}{b_3}
    & (a_3, b_3, \pm d_3) = gcd(s, \pm p, o)
\end{align*}

\subsection{$D_1 \times D_2 = F_3$}
\begin{align*}
D_1 \times D_2 &= \frac{\pm a_1\sqrt{c_1} \pm d_1 }{b_1} \times \frac{\pm a_2\sqrt{c_2} \pm d_2 }{b_2}\\
 &= \frac{\pm a_1a_2\sqrt{c_1c_2} \pm a_1d_2\sqrt{c1} \pm a_2d_1\sqrt{c_2} \pm d_1d_2}{b_1b_2}\\
\end{align*}

\subsection{MulDD $D_1 \times D_2 \mapsto H$ ???}
\begin{align*}
D_1 \times D_2 &= \frac{\pm b_1 \pm c_1\sqrt{d_1}}{a_1} \times \frac{\pm b_2 \pm c_2\sqrt{d_2}}{a_2}\\
 &= \frac{\pm b_1b_2 \pm b_1c_2\sqrt{d_2} \pm b_2c_1\sqrt{d_1} \pm c_1c_2\sqrt{d_1d_2} }{a_1a_2}\\
 &= \frac{\pm a_1a_2m\sqrt{c_3}}{b_1b_2} & c_1c_2 = m^2c_3\\
 &= \frac{\pm a_3\sqrt{c_3}}{b_3} & (\pm a_3,b_3) = gcd(\pm a_1a_2m,b_1b_2)
\end{align*}






















\end{document}